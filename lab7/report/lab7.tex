\documentclass[a4paper,14pt]{extarticle}

\usepackage[utf8x]{inputenc}
\usepackage[T1]{fontenc}
\usepackage[russian]{babel}
\usepackage{hyperref}
\usepackage{indentfirst}
\usepackage{here}
\usepackage{array}
\usepackage{graphicx}
\usepackage{grffile}
\usepackage{caption}
\usepackage{subcaption}
\usepackage{chngcntr}
\usepackage{amsmath}
\usepackage{amssymb}
\usepackage[left=2cm,right=2cm,top=2cm,bottom=2cm,bindingoffset=0cm]{geometry}
\usepackage{multicol}
\usepackage{multirow}
\usepackage{titlesec}
\usepackage{listings}
\usepackage{listingsutf8}
\usepackage{color}
\usepackage{enumitem}
\usepackage{cmap}
\usepackage{titlesec}

\definecolor{green}{rgb}{0,0.6,0}
\definecolor{gray}{rgb}{0.5,0.5,0.5}
\definecolor{purple}{rgb}{0.58,0,0.82}

\lstdefinelanguage{none}{}

\lstset{
	language={C++},
	inputpath={../},
	backgroundcolor=\color{white},
	commentstyle=\color{green},
	keywordstyle=\color{blue},
	numberstyle=\color{gray}\scriptsize\ttfamily,
	stringstyle=\color{purple},
	basicstyle=\lst@ifdisplaystyle\footnotesize\fi\ttfamily,
	breakatwhitespace=false,
	breaklines=true,
	captionpos=b,
	keepspaces=true,
	numbers=left,
	numbersep=5pt,
	showspaces=false,
	showstringspaces=false,
	showtabs=false,
	tabsize=4,
	frame=single,
	morekeywords={NULL, DWORD, WINAPI, HANDLE, STARTUPINFO, BYTE, LPSTR, SOCKET, WSADATA, TCHAR, LPCTSTR, LPOVERLAPPED, WSABUF, SECURITY_ATTRIBUTES, SECURITY_DESCRIPTOR, TRUE, FALSE, PROCESS_INFORMATION, PIPE_UNLIMITED_INSTANCES, LPVOID, sockaddr_in},
	deletekeywords={error},
	alsoletter={_},
	sensitive=true,
	extendedchars=false,
	columns=fullflexible,
	inputencoding=utf8/cp1251,
	literate=%
		{~}{{\raise.25ex\hbox{$\mathtt{\sim}$}}}{1}%
		{-}{-}{1}
}

\makeatletter
\def\lst@outputspace{{\ }}
\makeatother

\renewcommand{\le}{\ensuremath{\leqslant}}
\renewcommand{\leq}{\ensuremath{\leqslant}}
\renewcommand{\ge}{\ensuremath{\geqslant}}
\renewcommand{\geq}{\ensuremath{\geqslant}}
\renewcommand{\epsilon}{\ensuremath{\varepsilon}}
\renewcommand{\phi}{\ensuremath{\varphi}}
\renewcommand{\thefigure}{\arabic{figure}}
\newcommand{\code}[1]{\lstinline|#1|}
\newcommand{\caret}{\^{}}
\newcommand{\ctrl}[1]{\^{}{#1}}
\newcommand{\listingwithoutput}[1]{
	\lstinputlisting[caption=\code{#1.cpp}]{src/#1/#1.cpp}
	Выполним программу \code{#1.exe}:
	\lstinputlisting[language=none]{logs/#1/#1.txt}
}

\titleformat*{\section}{\large\bfseries}
\titleformat*{\subsection}{\normalsize\bfseries}
\titleformat*{\subsubsection}{\normalsize\bfseries}
\titleformat*{\paragraph}{\normalsize\bfseries}
\titleformat*{\subparagraph}{\normalsize\bfseries}

\titlespacing{\section}{0em}{0.8em}{0.8em}

\counterwithin{figure}{section}
\counterwithin{equation}{section}
\counterwithin{table}{section}
\newcommand{\sign}[1][5cm]{\makebox[#1]{\hrulefill}}
\newcommand{\equipollence}{\quad\Leftrightarrow\quad}
\newcommand{\no}[1]{\overline{#1}}
\graphicspath{{../pics/}}
\captionsetup{justification=centering,margin=1cm}
\def\arraystretch{1.3}
\setlength\parindent{5ex}
\titlelabel{\thetitle.\quad}

\setitemize{topsep=0em, itemsep=0em}
\setenumerate{topsep=0em, itemsep=0em}

\begin{document}

\begin{titlepage}
\begin{center}
	Санкт-Петербургский Политехнический Университет Петра Великого\\[0.3cm]
	Институт компьютерных наук и технологий \\[0.3cm]
	Кафедра компьютерных систем и программных технологий\\[4cm]
	
	\textbf{ОТЧЕТ}\\ 
	\textbf{по лабораторной работе}\\[0.5cm]
	\textbf{<<Процессы и потоки в Windows>>}\\[0.1cm]
	Операционные системы\\[3.0cm]
\end{center}

\begin{flushright}
	\begin{minipage}{0.5\textwidth}
		\textbf{Работу выполнил студент}\\[3mm]
		группа 43501/3 \hfill Дьячков В.В.\\[5mm]
		\textbf{Работу принял преподаватель}\\[5mm]
		\sign[2cm] \hfill к.т.н., доц. Душутина Е.В. \\[5mm]
	\end{minipage}
\end{flushright}

\vfill

\begin{center}
	Санкт-Петербург\\[0.3cm]
	\the\year
\end{center}
\end{titlepage}

\addtocounter{page}{1}

\tableofcontents
\newpage

\section{Цели работы}

Изучение видов межпроцессного взаимодействия в ОС Windows.

\section{Программа работы}

\renewcommand{\labelenumii}{\theenumii}
\renewcommand{\theenumii}{\theenumi.\arabic{enumii}.}

\textbf{Порождение и запуск процессов}

\begin{enumerate}
	\item Неименованные каналы (pipe):
		\begin{enumerate}
			\item Создать клиент-серверное приложение, позволяющее набираемые символы в терминальном окне командной строки (сервер) отображать их в окно процесса-потомка (клиент).
			\item Создать эхо-сервер, взаимодействующий с клиентом посредством pipe.
		\end{enumerate}
	\item Именованные каналы (named pipe):
		\begin{enumerate}
			\item Реализовать между одним клиентом и сервером обмен данными, вводимыми с консоли на стороне клиента и возвращаемыми сервером обратно до получения команды exit.
			\item Реализовать между сервером и множеством клиентов обмен данными, вводимыми с консоли на стороне клиента и возвращаемыми сервером обратно до получения команды exit.
			\item Модифицировать приложение из предыдущего примера для сетевого обмена информацией.
		\end{enumerate}
	\item Сокеты (socket):
		\begin{enumerate}
			\item Реализовать программ локального и сетевую обмена с помощью сокетов с использованием потокового протокола с установлением соединения (TCP в стеке TCP/IP).
			\item Модифицировать программу для локального обмена с множеством клиентов и с доступом к общему ресурсу. Провести эксперимент с множеством клиентов при сетевом обмене, представить результаты для виртуальной и реальной сетей.
			\item Проанализировать пример применения сокетов (сетевой обмен «мгновенными» сообщениями).
			\item Привести примеры использования портов завершения. Привести пример приложения с большим количеством клиентов до 1000 (когда порты завершения оправданы), общее количество потоков не более 10.
			\item Реализовать обмен на основе UDP.
		\end{enumerate}
	\item Сигналы (signal):
		\begin{enumerate}
			\item Задать обработчик сигналов завершения для консольного приложения.
			\item Самостоятельно предложить собственную реализацию обработчика сигнала.
		\end{enumerate}
	\item Разделяемая память (file mapping):
		\begin{enumerate}
			\item Создать программу, в которой первый процесс генерирует случайное число и записывает его в буфер, доступный второму процессу, откуда он его и считывает с последующим выводом.
		\end{enumerate}
	\item Почтовые слоты (MailSlot):
		\begin{enumerate}
			\item Предложить собственную реализацию приложения, иллюстрирующую обмен информацией почтовыми слотами. Продемонстрировать возможность локального и удаленного доступа. Выполнить широковещательную передачу данных.
		\end{enumerate}
\end{enumerate}

\section{Используемое окружение}

\begin{itemize}
	\item ОС: Windows 10 Pro
	\item Версия ОС: 1803 (сборка 17134.1069)
	\item Процессор: Intel® Core™ i7-8550U CPU @ 1.80GHz × 8
	\item ОЗУ: 16 Гб
	\item Компилятор: MSVC 14.23.28105, C/C++ Optimizing Compiler Version 19.23.28105.4 for x64
\end{itemize}

\section{Функции для работы с исключениями}

\lstinputlisting[caption=\code{ExceptionHandlingUtils.h}]{src/ExceptionHandling/ExceptionHandlingUtils.h}

\lstinputlisting[caption=\code{ExceptionHandlingUtils.cpp}]{src/ExceptionHandling/ExceptionHandlingUtils.cpp}

\section{Ошибка деления на ноль}

\subsection{Обработка исключения при помощи функций WinAPI}

\listingwithoutput{DBZWinAPI}

\subsection{Обработка исключения при помощи функции-фильтра}

\listingwithoutput{DBZFilter}

\subsection{Генерация исключения при помощи \code{RaiseException}}

\listingwithoutput{DBZRaise}

\subsection{Функции для обработки необработанных исключений}

\listingwithoutput{DBZUnhandle}

\subsection{Обработка вложенных исключений}

\listingwithoutput{DBZNested}

\subsection{Использование операторов \code{goto} и \code{__leave} для выхода из блока \code{__try}}
\label{sec:gotoleave}

\listingwithoutput{DBZGotoLeave}

\subsection{Преобразование в исключение языка \code{C++}}

\listingwithoutput{DBZTranslator}

\subsection{Использование финального обработчика \code{__finally}}

\listingwithoutput{DBZFinally}

\subsection{Использование \code{AbnormalTermination} в блоке \code{__finally}}

\listingwithoutput{DBZAbnormalTermination}

\newpage

\section{Выполнение несуществующий операции}

\subsection{Обработка исключения при помощи функций WinAPI}

\listingwithoutput{IIWinAPI}

\subsection{Обработка исключения при помощи функции-фильтра}

\listingwithoutput{IIFilter}

\subsection{Генерация исключения при помощи \code{RaiseException}}

\listingwithoutput{IIRaise}

\subsection{Функции для обработки необработанных исключений}

\listingwithoutput{IIUnhandle}

\subsection{Обработка вложенных исключений}

\listingwithoutput{IINested}

\subsection{Использование операторов \code{goto} и \code{__leave} для выхода из блока \code{__try}}

Аналогично разделу с ошибкой деления на ноль (см. \ref{sec:gotoleave}).

\subsection{Преобразование в исключение языка \code{C++}}

\listingwithoutput{IITranslator}

\subsection{Использование финального обработчика \code{__finally}}

\listingwithoutput{IIFinally}

\subsection{Использование \code{AbnormalTermination} в блоке \code{__finally}}

\listingwithoutput{IIAbnormalTermination}

\newpage

\section{Выводы}

В процессе выполнения данной работы:

\begin{itemize}
	\item 1
	\item 2
\end{itemize}

\section*{Список использованных источников}

\begin{enumerate}
	\item Душутина Е.В. -- Системное программное обеспечение. Практические вопросы разработки системных приложений [Текст] -- 2016.
	\item Таненбаум Э. -- Современные операционные системы [Текст] -- 2015.
	\item Structured Exception Handling / Microsoft Docs [Электронный ресурс]:\\
		{\small\url{https://docs.microsoft.com/en-us/cpp/cpp/structured-exception-handling-c-cpp}}
	\item \_\_ud2 / Microsoft Docs [Электронный ресурс]:\\
		{\small\url{https://docs.microsoft.com/en-us/cpp/intrinsics/ud2}}
\end{enumerate}

\end{document}
