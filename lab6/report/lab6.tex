\documentclass[a4paper,14pt]{extarticle}

\usepackage[utf8x]{inputenc}
\usepackage[T1]{fontenc}
\usepackage[russian]{babel}
\usepackage{hyperref}
\usepackage{indentfirst}
\usepackage{here}
\usepackage{array}
\usepackage{graphicx}
\usepackage{grffile}
\usepackage{caption}
\usepackage{subcaption}
\usepackage{chngcntr}
\usepackage{amsmath}
\usepackage{amssymb}
\usepackage{pgfplots}
\usepackage{pgfplotstable}
\usepackage[left=2cm,right=2cm,top=2cm,bottom=2cm,bindingoffset=0cm]{geometry}
\usepackage{multicol}
\usepackage{multirow}
\usepackage{titlesec}
\usepackage{listings}
\usepackage{color}
\usepackage{longtable}
\usepackage{enumitem}
\usepackage{cmap}

\usetikzlibrary{shapes,arrows}

\definecolor{green}{rgb}{0,0.6,0}
\definecolor{gray}{rgb}{0.5,0.5,0.5}
\definecolor{purple}{rgb}{0.58,0,0.82}

\lstdefinelanguage{none}{}

\lstset{
	language={bash},
	inputpath={../logs/},
	backgroundcolor=\color{white},
	commentstyle=\color{green},
	keywordstyle=\color{blue},
	numberstyle=\scriptsize\color{gray},
	stringstyle=\color{purple},
	basicstyle=\small,
	breakatwhitespace=false,
	breaklines=true,
	captionpos=b,
	keepspaces=true,
	numbers=left,
	numbersep=5pt,
	showspaces=false,
	showstringspaces=false,
	showtabs=false,
	tabsize=4,
	texcl=true,
	frame=single,
	morekeywords={user@SPOComp13, vaddya@turing, date, who, tty, logname, uname, sleep, man, ls, mv, cp, rm, mkdir, rmdir, grep, diff, sort, tr, cut, cmp, cc},
	deletekeywords={test},
	literate=%
		{~}{{\raise.25ex\hbox{$\mathtt{\sim}$}}}{1}%
		{-}{-}{1}
}

\renewcommand{\le}{\ensuremath{\leqslant}}
\renewcommand{\leq}{\ensuremath{\leqslant}}
\renewcommand{\ge}{\ensuremath{\geqslant}}
\renewcommand{\geq}{\ensuremath{\geqslant}}
\renewcommand{\epsilon}{\ensuremath{\varepsilon}}
\renewcommand{\phi}{\ensuremath{\varphi}}
\renewcommand{\thefigure}{\arabic{figure}}
\def\code#1{\texttt{#1}}
\newcommand{\caret}{\^{}}

\titleformat*{\section}{\large\bfseries} 
\titleformat*{\subsection}{\normalsize\bfseries} 
\titleformat*{\subsubsection}{\normalsize\bfseries} 
\titleformat*{\paragraph}{\normalsize\bfseries} 
\titleformat*{\subparagraph}{\normalsize\bfseries} 

\counterwithin{figure}{section}
\counterwithin{equation}{section}
\counterwithin{table}{section}
\newcommand{\sign}[1][5cm]{\makebox[#1]{\hrulefill}}
\newcommand{\equipollence}{\quad\Leftrightarrow\quad}
\newcommand{\no}[1]{\overline{#1}}
\graphicspath{{../pics/}{../screens/}}
\captionsetup{justification=centering,margin=1cm}
\def\arraystretch{1.3}
\setlength\parindent{5ex}
\titlelabel{\thetitle.\quad}

\setitemize{topsep=0.5em, itemsep=0em}
\setenumerate{topsep=0.5em, itemsep=0em}

\begin{document}

\begin{titlepage}
\begin{center}
	Санкт-Петербургский Политехнический Университет Петра Великого\\[0.3cm]
	Институт компьютерных наук и технологий \\[0.3cm]
	Кафедра компьютерных систем и программных технологий\\[4cm]
	
	\textbf{ОТЧЕТ}\\ 
	\textbf{по лабораторной работе}\\[0.5cm]
	\textbf{<<Интерпретаторы командной строки ОС Linux>>}\\[0.1cm]
	Операционные системы\\[3.0cm]
\end{center}

\begin{flushright}
	\begin{minipage}{0.5\textwidth}
		\textbf{Работу выполнил студент}\\[3mm]
		группа 43501/3 \hfill Дьячков В.В.\\[5mm]
		\textbf{Работу принял преподаватель}\\[5mm]
		\sign[2cm] \hfill к.т.н., доц. Душутина Е.В. \\[5mm]
	\end{minipage}
\end{flushright}

\vfill

\begin{center}
	Санкт-Петербург\\[0.3cm]
	\the\year
\end{center}
\end{titlepage}

\addtocounter{page}{1}

\tableofcontents
\newpage

\section{Цели работы}

Изучение управления процессами и потоками в Windows.

\section{Программа работы}

\renewcommand{\labelenumii}{\theenumii}
\renewcommand{\theenumii}{\theenumi.\arabic{enumii}.}

\textbf{Порождение и запуск процессов}

\begin{enumerate}
	\item Исследуйте результаты работы программы 3.1 и 3.2 в зависимости от того, какой приоритет назначается базовому потоку: \code{IDLE}, \code{BELOW\_NORMAL}, \code{NORMAL}, \code{ABODE\_NORMAL}, \code{HIGH}, \code{REALTIME};
	\item Модифицируйте программу 3.2 для заполнения таблицы 2 текущими данными вашего эксперимента. Сделайте выводы;
	\item Проанализируйте работу программы из примера 3.3.
	\item С помощью утилит CPU Stress, позволяющих нагружать систему, и утилиты мониторинга \code{ProcessExplorer()} (или иных утилит) зафиксируйте динамическое изменение приоритетов, приведите результаты в отчете;
	\item Создайте программу, демонстрирующую возможность наследования:
		\begin{enumerate}
			\item дескриптора порождающего процесса,
			\item дескрипторов открытых файлов,
		\end{enumerate}
	для выполнения этого задания следует учесть, что по умолчанию наследование в Windows откючено и для возможности наследования, необходимо:
		\begin{itemize}
			\item разрешить процессу-потомку наследовать дескрипторы,
			\item сделать дескрипторы наследуемыми.
		\end{itemize}
\end{enumerate}

\section{Используемая операционная система}

\begin{itemize}
	\item ОС: Windows 10 Pro
	\item Версия ОС: 1803 (сборка 17134.345)
	\item Процессор: Intel® Core™ i7-4800MQ CPU @ 2.70GHz × 8
\end{itemize}

\noindent Информация о сети (вывод утилиты \code{ipconfig})

\lstinputlisting[language=none]{logs/ipconfig.txt}

\newpage

\section{Неименованные каналы}

\paragraph{Задание.} Создать клиент-серверное приложение, позволяющее набираемые символы в терминальном окне командной строки (сервер) отображать их в окно процесса-потомка (клиент).

\lstinputlisting[caption=\code{pipe_terminal_master.cpp}]{src/pipe/terminal/master/master.cpp}

\lstinputlisting[caption=\code{pipe_terminal_slave.cpp}]{src/pipe/terminal/slave/slave.cpp}

Запустим \code{pipe_terminal_master.exe}:

\lstinputlisting[language=none]{logs/pipe/terminal/master.txt}

\lstinputlisting[language=none]{logs/pipe/terminal/slave.txt}

Видно, что запуске сервера создается канал, после чего порождается процесс \code{pipe_terminal_slave.exe} в новом окне. Любые символы, которые пишем в окне сервера, моментально появляются в окне клиента. Это достигается за счет того, что при порождении процесса в структуре \code{STARTUPINFO} поле \code{hStdInput} было заменено на читающий дескриптор канала.

\paragraph{Задание.} Создать эхо-сервер, взаимодействующий с клиентом посредством pipe.

В программе используется передача дескрипторов через наследование. По причине того, что анонимный канал является полудуплексным, для организации эхо-сервера необходимо создавать 2 канала (для передачи от клиента-серверу и обратно). При этом ненужные дескрипторы каналов закрываются только на стороне сервера (т.к. клиент наследует 4 дескриптора, а явно мы передаем только 2 дескриптора).

\lstinputlisting[caption=\code{pipe_echo_server.cpp}]{src/pipe/echo/server/server.cpp}

\lstinputlisting[caption=\code{pipe_echo_client.cpp}]{src/pipe/echo/client/client.cpp}

Запустим \code{pipe_echo_server.exe}:

\lstinputlisting[language=none]{logs/pipe/echo/server.txt}

\lstinputlisting[language=none]{logs/pipe/echo/client.txt}

Сервер создает два набора каналов чтобы обеспечить дуплексный обмен данными. После этого порождает процесс \code{pipe_echo_client.exe} и ожидает запрос от клиента. Приняв запрос, сервер фиксирует событие и отправляет это же сообщение обратно клиенту. Обмен происходит 10 раз, после чего процесс завершается.

\section{Именованные каналы}

Именованные каналы являются дуплексными, ориентированы на обмен сообщениями и обеспечивают взаимодействие через сеть. Кроме того, один именованный канал может иметь несколько открытых дескрипторов. В сочетании с удобными, ориентированными на выполнение транзакций функциями эти возможности делают именованные каналы пригодными для создания клиент-серверных систем. Обмен данными может быть синхронным и асинхронным.

\paragraph{Задание.} Реализовать между одним клиентом и сервером обмен данными, вводимыми с консоли на стороне клиента и возвращаемыми сервером обратно до получения команды exit.

Программа-сервер (\code{named_pipe_echo_server}) создает именованый канал для двунаправленного использования и ожидает подключения программы-клиента (\code{named_pipe_terminal_slave}).

\lstinputlisting[caption=\code{named_pipe_echo_server.cpp}]{src/named_pipe/echo/server/server.cpp}

\lstinputlisting[caption=\code{named_pipe_echo_client.cpp}]{src/named_pipe/echo/client/client.cpp}

Запустим сначала программу-сервер, а затем с помощью программы-клиента подключимся к созданному каналу и отправим несколько сообщений.

\lstinputlisting[language=none]{logs/named_pipe/echo/server.txt}

\lstinputlisting[language=none]{logs/named_pipe/echo/client.txt}

Из результатов видно, что сервер успешно создал канал, к которому после этого подключился клиент. Клиент отправляет сообщения в именованный канал, а сервер читает их оттуда и отправляет обратно. При получении команды exit сервер завершает свою работу.

\paragraph{Задание.} Реализовать между сервером и множеством клиентов обмен данными, вводимыми с консоли на стороне клиента и возвращаемыми сервером обратно до получения команды exit. Обеспечить возможность сетевого обмена информацией.

Для того, чтобы реализовать возможность обмена данными сервером с несколькими клиентами, воспользуемся механизмом потоков. В цикле будет создаваться экземпляр именованного канала, после чего ожидать подключения клиента. После подключения очередного клиента будет создан поток для взаимодействия с клиентом. Основная программа перейдет к ожиданию нового клиента.

Для возможности сетевого обмена предусмотрим в клиентской программе возможность задания IP-адреса сервера с помощью параметром командной строки (при отсутствии параметров клиент будет подключаться к локальному серверу).

\lstinputlisting[caption=\code{named_pipe_multiecho_server.cpp}]{src/named_pipe/multiecho/server/server.cpp}

\lstinputlisting[caption=\code{named_pipe_multiecho_client.cpp}]{src/named_pipe/multiecho/client/client.cpp}

%Для тестирования взаимодействия по сети будем использовать виртуальную машину с Windows 10. Выведем на ней информацию о сети:

%\lstinputlisting[language=none]{logs/named_pipe/multiecho/ipconfig.txt}

\section{Сокеты}

\paragraph{Задание.} Реализовать программа локального обмена сокетами с использованием потокового протокола с установлением соединения (TCP в стеке TCP/IP). Модифицировать программу для локального обмена с множеством клиентов и с доступом к общему ресурсу.

\paragraph{Задание.}  Реализовать сетевую передача данных с помощью сокетов. В код программ клиента и сервера необходимо внести незначительные изменения, поскольку сокеты изначально предназначены для удаленного взаимодействия. Провести эксперимент с множеством клиентов при сетевом обмене, представить результаты для виртуальной и реальной сетей.

\paragraph{Задание.} Проанализировать пример применения сокетов (сетевой обмен «мгновенными» сообщениями).

\paragraph{Задание.} Привести примеры использования портов завершения. Привести пример приложения с большим количеством клиентов до 1000 (когда порты завершения оправданы), общее количество потоков не более 10.

\paragraph{Задание.} Реализовать обмен на основе UDP.

\section{Сигналы}

\paragraph{Задание.} Задать обработчик сигналов завершения для консольного приложения.

\paragraph{Задание.} Самостоятельно предложить собственную реализацию обработчика сигнала.

\section{Разделяемая память}

\paragraph{Задание.} Создать программу, в которой первый процесс генерирует случайное число и записывает его в буфер, доступный второму процессу, откуда он его и считывает с последующим выводом.

\section{Почтовые слоты}

\paragraph{Задание.} Предложить собственную реализацию приложения, иллюстрирующую обмен информацией почтовыми слотами.

\paragraph{Задание.} Продемонстрировать возможность локального и удаленного доступа.

\paragraph{Задание.} Выполнить широковещательную передачу данных.

\section{Выводы}

В процессе выполнения данной работы:

\begin{itemize}
	\item изучены различные типы средства межпроцессорного взаимодействия в Windows;
	\item рассмотрены возможности неименованных и именованных каналов;
	\item продемонстрированы возможности использования сигналов для консольного приложения; 
	\item рассмотрены почтовые слоты и разделяемая память как вид межпроцессного взаимодействия.
\end{itemize}

\section*{Список использованных источников}

\begin{enumerate}
	\item Душутина Е.В. - Системное программное обеспечение. Практические вопросы разработки системных приложений [Текст] -- 2016.
	\item Таненбаум Э. - Современные операционные системы [Текст] -- 2015.
	\item Named Pipe Instances / Microsoft Docs [Электронный ресурс]:\\
		{\small\url{https://docs.microsoft.com/en-us/windows/desktop/ipc/named-pipe-instances}} 
	\item Multithreaded Pipe Server / Microsoft Docs [Электронный ресурс]:\\
		{\small\url{https://docs.microsoft.com/en-us/windows/desktop/ipc/multithreaded-pipe-server}} 
\end{enumerate}

\end{document}
