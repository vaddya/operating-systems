\documentclass[a4paper,14pt]{extarticle}

\usepackage[utf8x]{inputenc}
\usepackage[T1]{fontenc}
\usepackage[russian]{babel}
\usepackage{hyperref}
\usepackage{indentfirst}
\usepackage{here}
\usepackage{array}
\usepackage{graphicx}
\usepackage{grffile}
\usepackage{caption}
\usepackage{subcaption}
\usepackage{chngcntr}
\usepackage{amsmath}
\usepackage{amssymb}
\usepackage{pgfplots}
\usepackage{pgfplotstable}
\usepackage[left=2cm,right=2cm,top=2cm,bottom=2cm,bindingoffset=0cm]{geometry}
\usepackage{multicol}
\usepackage{multirow}
\usepackage{titlesec}
\usepackage{listings}
\usepackage{color}
\usepackage{longtable}
\usepackage{enumitem}
\usepackage{cmap}

\usetikzlibrary{shapes,arrows}

\definecolor{green}{rgb}{0,0.6,0}
\definecolor{gray}{rgb}{0.5,0.5,0.5}
\definecolor{purple}{rgb}{0.58,0,0.82}

\lstdefinelanguage{none}{}

\lstset{
	language={bash},
	inputpath={../logs/},
	backgroundcolor=\color{white},
	commentstyle=\color{green},
	keywordstyle=\color{blue},
	numberstyle=\scriptsize\color{gray},
	stringstyle=\color{purple},
	basicstyle=\small,
	breakatwhitespace=false,
	breaklines=true,
	captionpos=b,
	keepspaces=true,
	numbers=left,
	numbersep=5pt,
	showspaces=false,
	showstringspaces=false,
	showtabs=false,
	tabsize=4,
	texcl=true,
	frame=single,
	morekeywords={user@SPOComp13, vaddya@turing, date, who, tty, logname, uname, sleep, man, ls, mv, cp, rm, mkdir, rmdir, grep, diff, sort, tr, cut, cmp, cc},
	deletekeywords={test},
	literate=%
		{~}{{\raise.25ex\hbox{$\mathtt{\sim}$}}}{1}%
		{-}{-}{1}
}

\renewcommand{\le}{\ensuremath{\leqslant}}
\renewcommand{\leq}{\ensuremath{\leqslant}}
\renewcommand{\ge}{\ensuremath{\geqslant}}
\renewcommand{\geq}{\ensuremath{\geqslant}}
\renewcommand{\epsilon}{\ensuremath{\varepsilon}}
\renewcommand{\phi}{\ensuremath{\varphi}}
\renewcommand{\thefigure}{\arabic{figure}}
\def\code#1{\texttt{#1}}
\newcommand{\caret}{\^{}}

\titleformat*{\section}{\large\bfseries} 
\titleformat*{\subsection}{\normalsize\bfseries} 
\titleformat*{\subsubsection}{\normalsize\bfseries} 
\titleformat*{\paragraph}{\normalsize\bfseries} 
\titleformat*{\subparagraph}{\normalsize\bfseries} 

\counterwithin{figure}{section}
\counterwithin{equation}{section}
\counterwithin{table}{section}
\newcommand{\sign}[1][5cm]{\makebox[#1]{\hrulefill}}
\newcommand{\equipollence}{\quad\Leftrightarrow\quad}
\newcommand{\no}[1]{\overline{#1}}
\graphicspath{{../pics/}{../screens/}}
\captionsetup{justification=centering,margin=1cm}
\def\arraystretch{1.3}
\setlength\parindent{5ex}
\titlelabel{\thetitle.\quad}

\setitemize{topsep=0.5em, itemsep=0em}
\setenumerate{topsep=0.5em, itemsep=0em}

\begin{document}

\begin{titlepage}
\begin{center}
	Санкт-Петербургский Политехнический Университет Петра Великого\\[0.3cm]
	Институт компьютерных наук и технологий \\[0.3cm]
	Кафедра компьютерных систем и программных технологий\\[4cm]
	
	\textbf{ОТЧЕТ}\\ 
	\textbf{по лабораторной работе}\\[0.5cm]
	\textbf{<<Интерпретаторы командной строки ОС Linux>>}\\[0.1cm]
	Операционные системы\\[3.0cm]
\end{center}

\begin{flushright}
	\begin{minipage}{0.5\textwidth}
		\textbf{Работу выполнил студент}\\[3mm]
		группа 43501/3 \hfill Дьячков В.В.\\[5mm]
		\textbf{Работу принял преподаватель}\\[5mm]
		\sign[2cm] \hfill к.т.н., доц. Душутина Е.В. \\[5mm]
	\end{minipage}
\end{flushright}

\vfill

\begin{center}
	Санкт-Петербург\\[0.3cm]
	\the\year
\end{center}
\end{titlepage}

\addtocounter{page}{1}

\tableofcontents
\newpage

\section{Цели работы}

Изучение механизмов перехвата вызовов системных функций в ОС Windows.

\section{Программа работы}

\renewcommand{\labelenumii}{\theenumii}
\renewcommand{\theenumii}{\theenumi.\arabic{enumii}.}

\textbf{Порождение и запуск процессов}

\begin{enumerate}
	\item Исследуйте результаты работы программы 3.1 и 3.2 в зависимости от того, какой приоритет назначается базовому потоку: \code{IDLE}, \code{BELOW\_NORMAL}, \code{NORMAL}, \code{ABODE\_NORMAL}, \code{HIGH}, \code{REALTIME};
	\item Модифицируйте программу 3.2 для заполнения таблицы 2 текущими данными вашего эксперимента. Сделайте выводы;
	\item Проанализируйте работу программы из примера 3.3.
	\item С помощью утилит CPU Stress, позволяющих нагружать систему, и утилиты мониторинга \code{ProcessExplorer()} (или иных утилит) зафиксируйте динамическое изменение приоритетов, приведите результаты в отчете;
	\item Создайте программу, демонстрирующую возможность наследования:
		\begin{enumerate}
			\item дескриптора порождающего процесса,
			\item дескрипторов открытых файлов,
		\end{enumerate}
	для выполнения этого задания следует учесть, что по умолчанию наследование в Windows откючено и для возможности наследования, необходимо:
		\begin{itemize}
			\item разрешить процессу-потомку наследовать дескрипторы,
			\item сделать дескрипторы наследуемыми.
		\end{itemize}
\end{enumerate}

\section{Используемое окружение}

\begin{itemize}
	\item ОС: Windows 10 Pro
	\item Версия ОС: 1803 (сборка 17134.1069)
	\item Процессор: Intel® Core™ i7-8550U CPU @ 1.80GHz × 8
	\item ОЗУ: 16 Гб
	\item Компилятор: MSVC 14.23.28105, C/C++ Optimizing Compiler Version 19.23.28105.4 for x64
\end{itemize}

%\section{Перехват системных вызовов}

%\section{Обзор библиотек, предназначенных для перехвата системных вызовов}

\section{Реализация индивидуального задания}

В качестве индивидуального задания необходимо реализовать программу перехвата обращения к устройству. Для перехвата используется библиотека Detours\footnote{\url{https://github.com/microsoft/Detours}}. 

\subsection{Программа обращения к устройству}

Для начала реализуем программу, которая получает и выводит информацию о размере диска, имя которого получает из передаваемых аргументов командной строки.

\lstinputlisting[caption=\code{DeviceIoControlApp.cpp}]{src/DeviceIoControlApp/DeviceIoControlApp.cpp}

Рассмотрим программу построчно:
\begin{itemize}
	\item [5-14] Функция, которая по передаваемому указателю на диск возвращает информацию о его конфигурации;
	\item [16-18] Функция, которая по информации о диске возвращает его размер в байтах;
	\item [22-30] Обработка входных параметров;
	\item [32-47] Открытие устройства и получения указателя на него;
	\item [49-68] Запрос информации о диске и вывод информации о нем.
\end{itemize}

Запустим программу и получим размер диска \code{C:}

\lstinputlisting[caption=\code{oringinal.txt}]{logs/original.txt}

Из вывода видно, что размер диска составляет 238 ГБ.

\subsection{Программа перехвата обращения к устройству}

Теперь при помощи библиотеки Detours реализуем перехват обращения к устройству. В случае успешного перехвата будем выводить информацию в консоль.

\lstinputlisting[caption=\code{HooksLib/dllmain.cpp}]{src/HooksLib/dllmain.cpp}

\lstinputlisting[caption=\code{HooksLib/export.def}]{src/HooksLib/export.def}

\subsection{Программа внедрения перехватчика в приложение}

Наконец реализуем программу, которая внедряет разработанную программу-перехватчик в приложение. Для этого также используются функциональные возможности Detours, а именно функция \code{DetoursCreateProcessWithDllA}.

\lstinputlisting[caption=\code{InjectDll.cpp}]{src/InjectDll/InjectDll.cpp}

В случае успеха, программа ожидает завершения порожденного процесса и выводит информацию о состоянии в консоль.

Запустим программу и получим размер диска \code{C:}

\lstinputlisting[caption=\code{hooked.txt}]{logs/hooked.txt}

Из вывода видно, что вызов системой функции был перехвачен при помощи разработанной программы и выедено соответствующее сообщение в консоль (строка 10). Изученный механизм можно использовать для построения более сложных программ.

\section{Выводы}

В процессе выполнения данной работы:

\begin{itemize}
	\item Изучена библиотека Detours, позволяющая перехватывать вызов системных функций;
	\item Реализован и продемонстрирован перехват вызова системной функции.
\end{itemize}

\newpage

\section*{Список использованных источников}

\begin{enumerate}
	\item Душутина Е.В. -- Системное программное обеспечение. Практические вопросы разработки системных приложений [Текст] -- 2016.
	\item Таненбаум Э. -- Современные операционные системы [Текст] -- 2015.
	\item Detours / Microsoft Research [Электронный ресурс]:\\
		{\small\url{https://www.microsoft.com/en-us/research/project/detours/}}
\end{enumerate}

\end{document}
