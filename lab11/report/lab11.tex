\documentclass[a4paper,14pt]{extarticle}

\usepackage[utf8x]{inputenc}
\usepackage[T1]{fontenc}
\usepackage[russian]{babel}
\usepackage{hyperref}
\usepackage{indentfirst}
\usepackage{here}
\usepackage{array}
\usepackage{graphicx}
\usepackage{grffile}
\usepackage{caption}
\usepackage{subcaption}
\usepackage{chngcntr}
\usepackage{amsmath}
\usepackage{amssymb}
\usepackage[left=2cm,right=2cm,top=2cm,bottom=2cm,bindingoffset=0cm]{geometry}
\usepackage{multicol}
\usepackage{multirow}
\usepackage{titlesec}
\usepackage{listings}
\usepackage{listingsutf8}
\usepackage{color}
\usepackage{enumitem}
\usepackage{cmap}
\usepackage{titlesec}

\definecolor{green}{rgb}{0,0.6,0}
\definecolor{gray}{rgb}{0.5,0.5,0.5}
\definecolor{purple}{rgb}{0.58,0,0.82}

\lstdefinelanguage{none}{}

\lstset{
	language={C++},
	inputpath={../},
	backgroundcolor=\color{white},
	commentstyle=\color{green},
	keywordstyle=\color{blue},
	numberstyle=\color{gray}\scriptsize\ttfamily,
	stringstyle=\color{purple},
	basicstyle=\lst@ifdisplaystyle\footnotesize\fi\ttfamily,
	breakatwhitespace=false,
	breaklines=true,
	captionpos=b,
	keepspaces=true,
	numbers=left,
	numbersep=5pt,
	showspaces=false,
	showstringspaces=false,
	showtabs=false,
	tabsize=4,
	frame=single,
	morekeywords={NULL, DWORD, WINAPI, HANDLE, STARTUPINFO, BYTE, LPSTR, SOCKET, WSADATA, TCHAR, LPCTSTR, LPOVERLAPPED, WSABUF, SECURITY_ATTRIBUTES, SECURITY_DESCRIPTOR, TRUE, FALSE, PROCESS_INFORMATION, PIPE_UNLIMITED_INSTANCES, LPVOID, sockaddr_in},
	deletekeywords={error},
	alsoletter={_},
	sensitive=true,
	extendedchars=false,
	columns=fullflexible,
	inputencoding=utf8/cp1251,
	literate=%
		{~}{{\raise.25ex\hbox{$\mathtt{\sim}$}}}{1}%
		{-}{-}{1}
}

\makeatletter
\def\lst@outputspace{{\ }}
\makeatother

\renewcommand{\le}{\ensuremath{\leqslant}}
\renewcommand{\leq}{\ensuremath{\leqslant}}
\renewcommand{\ge}{\ensuremath{\geqslant}}
\renewcommand{\geq}{\ensuremath{\geqslant}}
\renewcommand{\epsilon}{\ensuremath{\varepsilon}}
\renewcommand{\phi}{\ensuremath{\varphi}}
\renewcommand{\thefigure}{\arabic{figure}}
\newcommand{\code}[1]{\lstinline|#1|}
\newcommand{\caret}{\^{}}
\newcommand{\ctrl}[1]{\^{}{#1}}
\newcommand{\listingwithoutput}[1]{
	\lstinputlisting[caption=\code{#1.cpp}]{src/#1/#1.cpp}
	Выполним программу \code{#1.exe}:
	\lstinputlisting[language=none]{logs/#1/#1.txt}
}

\titleformat*{\section}{\large\bfseries}
\titleformat*{\subsection}{\normalsize\bfseries}
\titleformat*{\subsubsection}{\normalsize\bfseries}
\titleformat*{\paragraph}{\normalsize\bfseries}
\titleformat*{\subparagraph}{\normalsize\bfseries}

\titlespacing{\section}{0em}{0.8em}{0.8em}

\counterwithin{figure}{section}
\counterwithin{equation}{section}
\counterwithin{table}{section}
\newcommand{\sign}[1][5cm]{\makebox[#1]{\hrulefill}}
\newcommand{\equipollence}{\quad\Leftrightarrow\quad}
\newcommand{\no}[1]{\overline{#1}}
\graphicspath{{../pics/}}
\captionsetup{justification=centering,margin=1cm}
\def\arraystretch{1.3}
\setlength\parindent{5ex}
\titlelabel{\thetitle.\quad}

\setitemize{topsep=0em, itemsep=0em}
\setenumerate{topsep=0em, itemsep=0em}

\begin{document}

\begin{titlepage}
\begin{center}
	Санкт-Петербургский Политехнический Университет Петра Великого\\[0.3cm]
	Институт компьютерных наук и технологий \\[0.3cm]
	Кафедра компьютерных систем и программных технологий\\[4cm]
	
	\textbf{ОТЧЕТ}\\ 
	\textbf{по лабораторной работе}\\[0.5cm]
	\textbf{<<Процессы и потоки в Windows>>}\\[0.1cm]
	Операционные системы\\[3.0cm]
\end{center}

\begin{flushright}
	\begin{minipage}{0.5\textwidth}
		\textbf{Работу выполнил студент}\\[3mm]
		группа 43501/3 \hfill Дьячков В.В.\\[5mm]
		\textbf{Работу принял преподаватель}\\[5mm]
		\sign[2cm] \hfill к.т.н., доц. Душутина Е.В. \\[5mm]
	\end{minipage}
\end{flushright}

\vfill

\begin{center}
	Санкт-Петербург\\[0.3cm]
	\the\year
\end{center}
\end{titlepage}

\addtocounter{page}{1}

\tableofcontents
\newpage

\section{Цели работы}

Изучение работы ядра ОС Linux путем встраивания модуля драйвера символьного устройства в ядро операционной системы.

\section{Программа работы}

\renewcommand{\labelenumii}{\theenumii}
\renewcommand{\theenumii}{\theenumi.\arabic{enumii}.}

\textbf{Порождение и запуск процессов}

\begin{enumerate}
	\item Неименованные каналы (pipe):
		\begin{enumerate}
			\item Создать клиент-серверное приложение, позволяющее набираемые символы в терминальном окне командной строки (сервер) отображать их в окно процесса-потомка (клиент).
			\item Создать эхо-сервер, взаимодействующий с клиентом посредством pipe.
		\end{enumerate}
	\item Именованные каналы (named pipe):
		\begin{enumerate}
			\item Реализовать между одним клиентом и сервером обмен данными, вводимыми с консоли на стороне клиента и возвращаемыми сервером обратно до получения команды exit.
			\item Реализовать между сервером и множеством клиентов обмен данными, вводимыми с консоли на стороне клиента и возвращаемыми сервером обратно до получения команды exit.
			\item Модифицировать приложение из предыдущего примера для сетевого обмена информацией.
		\end{enumerate}
	\item Сокеты (socket):
		\begin{enumerate}
			\item Реализовать программ локального и сетевую обмена с помощью сокетов с использованием потокового протокола с установлением соединения (TCP в стеке TCP/IP).
			\item Модифицировать программу для локального обмена с множеством клиентов и с доступом к общему ресурсу. Провести эксперимент с множеством клиентов при сетевом обмене, представить результаты для виртуальной и реальной сетей.
			\item Проанализировать пример применения сокетов (сетевой обмен «мгновенными» сообщениями).
			\item Привести примеры использования портов завершения. Привести пример приложения с большим количеством клиентов до 1000 (когда порты завершения оправданы), общее количество потоков не более 10.
			\item Реализовать обмен на основе UDP.
		\end{enumerate}
	\item Сигналы (signal):
		\begin{enumerate}
			\item Задать обработчик сигналов завершения для консольного приложения.
			\item Самостоятельно предложить собственную реализацию обработчика сигнала.
		\end{enumerate}
	\item Разделяемая память (file mapping):
		\begin{enumerate}
			\item Создать программу, в которой первый процесс генерирует случайное число и записывает его в буфер, доступный второму процессу, откуда он его и считывает с последующим выводом.
		\end{enumerate}
	\item Почтовые слоты (MailSlot):
		\begin{enumerate}
			\item Предложить собственную реализацию приложения, иллюстрирующую обмен информацией почтовыми слотами. Продемонстрировать возможность локального и удаленного доступа. Выполнить широковещательную передачу данных.
		\end{enumerate}
\end{enumerate}

\section{Используемое окружение}

\begin{itemize}
	\item ОС: Ubuntu
	\item Версия ОС: 19.10
	\item Процессор: Intel® Core™ i7-8550U CPU @ 1.80GHz × 8
	\item ОЗУ: 16 Гб
	\item Компилятор: gcc (Ubuntu 9.2.1-9ubuntu2) 9.2.1 20191008
	\item Отладчик: GNU gdb (Ubuntu 8.3-0ubuntu1) 8.3
\end{itemize}

\section{Модуль ядра}

В данном модуле реализованны 6 функций:

\begin{itemize}
	\item \code{device_init} и \code{device_exit} будут вызваны при вставке и извлечении модуля из ядра;
	\item \code{device_open} и \code{device_release} будут вызваны при открытии и закрытии файла символьного устройства;
	\item \code{device_read} и \code{device_write} будут вызваны для обработки операций чтения и записи из устройства. В целях демонстрации (и упрощения реализации), считывание из файла будет происходит в обратном порядке, нежели чтение. То есть после записи слова <<Hello>>, считано будет слово <<olleH>>.
\end{itemize}

\lstinputlisting[caption=\code{driver.c}]{src/driver/driver.c}

Для модуля важно правильно описать \code{Makefile}. В первой строке указан специальный ключ \code{obj-m}. Цель \code{all} вызывает утилиту \code{make} для сборки кода модуля в рабочий драйвер с расширение \code{.ko}. Цели \code{insmod} и \code{rmmod} используются для встраивания и извлечения модуля из ядра. Цели \code{mknod} и \code{rmnod} используются для создания файла символьного устройства \code{/dev/chardev}, драйвером к которому и является разрабатываемый модуль. Вспомогательные цели \code{print} и \code{log} могут быть использованы для чтения системного журнала в целях отладки.

\lstinputlisting[caption=\code{Makefile}, language={[gnu] make}]{src/driver/Makefile}

Компиляция модуля осуществляется при помощи вызова утилиты \code{make} в директории с исходным кодом модуля.

\section{Обращение к символьному устройству}

Для демонстрации работы драйвера символьного устройства создадим программу, взаимодействующую с устройством через файл \code{/dev/chardev}.

\lstinputlisting[caption=\code{app.c}]{src/app/app.c}

Программа принимает в качестве аргумента строку, которую сначала записывает в символьное устройство, а зачем считывает из него же.
\code{Makefile} для этой программы содержит только одну цель для компиляции программы с помощью \code{gcc}.

\lstinputlisting[caption=\code{Makefile}, language={[gnu] make}]{src/app/Makefile}

\section{Тестирование работы модуля}

Проведем эксперимент с разработанным модулем и программой.

\logs{log}

Рассмотрим выполненные действия построчно:

\begin{enumerate}
	\item[$1-2$] Создание файла символьного устройства
	\item[$4-12$] Сборка модуля ядра
	\item[$14-15$] Встраивание модуля в ядро
	\item[$17-19$] Проверка, что встраивание прошло успешно
	\item[$21-27$] Запуск программы, обращающейся к символьному устройству. Видно, что файл был открыт, в него была записана строка <<Hello, driver!>>, затем было считано 14 байт -- перевернутая записанная строка, после чего файл был закрыт.
	\item[$29-37$] Журналирование работы драйвера
	\item[$39-49$] Извлечение модуля из ядра
	\item[$42-44$] После извлечения модуля, приложение не может произвести запись в символьный файл
	\item[$46-47$] Удаление файла символьного устройства
	\item[$49-51$] После удаления файла символьного устройства, программа не может его открыть
\end{enumerate}

Из вывода журналирования видно, что модуль успешно обработал запросы прикладной программы на открытие, запись, чтение и закрытие символьного устройства. После излечения модуля, запущенное прикладное приложение получило от ОС сигнал Killed (9).

\newpage

\section{Выводы}

В процессе выполнения данной работы:

\begin{itemize}
	\item разработан встраиваемый модуль для ядра ОС Linux; 
	\item разработана программа, взаимодействующая с <<устройством>> через символьный файл в директории \code{/dev};
	\item проведен эксперимент с встраиванием модуля в ядро и взаимодействием программы с файлом символьного устройства.
\end{itemize}

В результате описанных выше этапов был произведен полный цикл работы с модулем ядра ОС Linux.

\section*{Список использованных источников}

\begin{enumerate}
	\item Душутина Е.В. -- Системное программное обеспечение. Практические вопросы разработки системных приложений [Текст] -- 2016.
	\item Таненбаум Э. -- Современные операционные системы [Текст] -- 2015.
\end{enumerate}

\end{document}
