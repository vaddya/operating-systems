\documentclass[a4paper,14pt]{extarticle}

\usepackage[utf8x]{inputenc}
\usepackage[T1]{fontenc}
\usepackage[russian]{babel}
\usepackage{hyperref}
\usepackage{indentfirst}
\usepackage{here}
\usepackage{array}
\usepackage{graphicx}
\usepackage{grffile}
\usepackage{caption}
\usepackage{subcaption}
\usepackage{chngcntr}
\usepackage{amsmath}
\usepackage{amssymb}
\usepackage{pgfplots}
\usepackage{pgfplotstable}
\usepackage[left=2cm,right=2cm,top=2cm,bottom=2cm,bindingoffset=0cm]{geometry}
\usepackage{multicol}
\usepackage{multirow}
\usepackage{titlesec}
\usepackage{listings}
\usepackage{color}
\usepackage{longtable}
\usepackage{enumitem}
\usepackage{cmap}

\usetikzlibrary{shapes,arrows}

\definecolor{green}{rgb}{0,0.6,0}
\definecolor{gray}{rgb}{0.5,0.5,0.5}
\definecolor{purple}{rgb}{0.58,0,0.82}

\lstdefinelanguage{none}{}

\lstset{
	language={bash},
	inputpath={../logs/},
	backgroundcolor=\color{white},
	commentstyle=\color{green},
	keywordstyle=\color{blue},
	numberstyle=\scriptsize\color{gray},
	stringstyle=\color{purple},
	basicstyle=\small,
	breakatwhitespace=false,
	breaklines=true,
	captionpos=b,
	keepspaces=true,
	numbers=left,
	numbersep=5pt,
	showspaces=false,
	showstringspaces=false,
	showtabs=false,
	tabsize=4,
	texcl=true,
	frame=single,
	morekeywords={user@SPOComp13, vaddya@turing, date, who, tty, logname, uname, sleep, man, ls, mv, cp, rm, mkdir, rmdir, grep, diff, sort, tr, cut, cmp, cc},
	deletekeywords={test},
	literate=%
		{~}{{\raise.25ex\hbox{$\mathtt{\sim}$}}}{1}%
		{-}{-}{1}
}

\renewcommand{\le}{\ensuremath{\leqslant}}
\renewcommand{\leq}{\ensuremath{\leqslant}}
\renewcommand{\ge}{\ensuremath{\geqslant}}
\renewcommand{\geq}{\ensuremath{\geqslant}}
\renewcommand{\epsilon}{\ensuremath{\varepsilon}}
\renewcommand{\phi}{\ensuremath{\varphi}}
\renewcommand{\thefigure}{\arabic{figure}}
\def\code#1{\texttt{#1}}
\newcommand{\caret}{\^{}}

\titleformat*{\section}{\large\bfseries} 
\titleformat*{\subsection}{\normalsize\bfseries} 
\titleformat*{\subsubsection}{\normalsize\bfseries} 
\titleformat*{\paragraph}{\normalsize\bfseries} 
\titleformat*{\subparagraph}{\normalsize\bfseries} 

\counterwithin{figure}{section}
\counterwithin{equation}{section}
\counterwithin{table}{section}
\newcommand{\sign}[1][5cm]{\makebox[#1]{\hrulefill}}
\newcommand{\equipollence}{\quad\Leftrightarrow\quad}
\newcommand{\no}[1]{\overline{#1}}
\graphicspath{{../pics/}{../screens/}}
\captionsetup{justification=centering,margin=1cm}
\def\arraystretch{1.3}
\setlength\parindent{5ex}
\titlelabel{\thetitle.\quad}

\setitemize{topsep=0.5em, itemsep=0em}
\setenumerate{topsep=0.5em, itemsep=0em}

\begin{document}

\begin{titlepage}
\begin{center}
	Санкт-Петербургский Политехнический Университет Петра Великого\\[0.3cm]
	Институт компьютерных наук и технологий \\[0.3cm]
	Кафедра компьютерных систем и программных технологий\\[4cm]
	
	\textbf{ОТЧЕТ}\\ 
	\textbf{по лабораторной работе}\\[0.5cm]
	\textbf{<<Интерпретаторы командной строки ОС Linux>>}\\[0.1cm]
	Операционные системы\\[3.0cm]
\end{center}

\begin{flushright}
	\begin{minipage}{0.5\textwidth}
		\textbf{Работу выполнил студент}\\[3mm]
		группа 43501/3 \hfill Дьячков В.В.\\[5mm]
		\textbf{Работу принял преподаватель}\\[5mm]
		\sign[2cm] \hfill к.т.н., доц. Душутина Е.В. \\[5mm]
	\end{minipage}
\end{flushright}

\vfill

\begin{center}
	Санкт-Петербург\\[0.3cm]
	\the\year
\end{center}
\end{titlepage}

\addtocounter{page}{1}

\tableofcontents
\newpage

\section{Цели работы}

Осуществить анализ механизмов обработки аппаратных и программных исключений, доступных в операционной системе Linux.

\section{Программа работы}

\renewcommand{\labelenumii}{\theenumii}
\renewcommand{\theenumii}{\theenumi.\arabic{enumii}.}

\textbf{Порождение и запуск процессов}

\begin{enumerate}
	\item Исследуйте результаты работы программы 3.1 и 3.2 в зависимости от того, какой приоритет назначается базовому потоку: \code{IDLE}, \code{BELOW\_NORMAL}, \code{NORMAL}, \code{ABODE\_NORMAL}, \code{HIGH}, \code{REALTIME};
	\item Модифицируйте программу 3.2 для заполнения таблицы 2 текущими данными вашего эксперимента. Сделайте выводы;
	\item Проанализируйте работу программы из примера 3.3.
	\item С помощью утилит CPU Stress, позволяющих нагружать систему, и утилиты мониторинга \code{ProcessExplorer()} (или иных утилит) зафиксируйте динамическое изменение приоритетов, приведите результаты в отчете;
	\item Создайте программу, демонстрирующую возможность наследования:
		\begin{enumerate}
			\item дескриптора порождающего процесса,
			\item дескрипторов открытых файлов,
		\end{enumerate}
	для выполнения этого задания следует учесть, что по умолчанию наследование в Windows откючено и для возможности наследования, необходимо:
		\begin{itemize}
			\item разрешить процессу-потомку наследовать дескрипторы,
			\item сделать дескрипторы наследуемыми.
		\end{itemize}
\end{enumerate}

\section{Используемое окружение}

\begin{itemize}
	\item ОС: Ubuntu
	\item Версия ОС: 19.10
	\item Процессор: Intel® Core™ i7-8550U CPU @ 1.80GHz × 8
	\item ОЗУ: 16 Гб
	\item Компилятор: g++ (Ubuntu 9.2.1-9ubuntu2) 9.2.1 20191008
	\item Отладчик: GNU gdb (Ubuntu 8.3-0ubuntu1) 8.3
\end{itemize}

\section{Используемые утилиты}

\subsection{GNU gdb}

%TODO про gdb

\subsection{strace}

%TODO про strace

\section{Обработка сигналов}

\subsection{Типы сигналов}

Сигналы в системе UNIX  используются для того, чтобы сообщить процессу о том, что возникло какое-либо событие или необходимо обработать исключительное состояние. 

%TODO табличка сигналов

\subsection{Генерация исключительных ситуаций}

Рассмотрим программу генерации сигнала выполнения ошибочной арифметической операции \code{SIGFPE}. Для этого поделим некоторое число на ноль.

\listingwithoutput{sigfpe}

Видно, что программа завершилась с кодом ошибки 136, что соответствует ошибочной арифметической операции (делении на 0 в данном случае).

Запустим программу \code{sigfpe} в отладчике \code{gdb}.

\logs{sigfpe/debug}

Из строки 13 видно, что произошла арифметическая ошибка и программа получила сигнал \code{SIGFPE}, после чего программа была завершена.

Рассмотрим программу генерации сигнала, генерируемого при попытке выполнить неправильно сформированную, несуществующую или привилегированную инструкцию \code{SIGILL}. Для этого будем использовать функцию \code{__buildin_trap()}, которую предоставляет компилятор \code{g++}. Внутри этой функции выполняется несуществующая инструкция для того, чтобы завершить программу ненормально (abnormal exit).

\listingwithoutput{sigill}

Запустим программу \code{sigill} в отладчике \code{gdb}.

\logs{sigill/debug}

Из строки 13 видно, что была выполнена неправильная инструкция и программа получила сигнала \code{SIGILL}, после чего программа была завершена.

\subsection{Назначение обработчика сигнала с помощью \code{signal}}

\listingwithoutput{sigfpe_signal}

Запустим программу \code{sigfpe_signal} в отладчике \code{gdb}.

\logs{sigfpe_signal/debug}

%TODO comments

\listingwithoutput{sigill_signal}

Запустим программу \code{sigill_signal} в отладчике \code{gdb}.

\logs{sigill_signal/debug}

%TODO comments

\subsection{Назначение обработчика сигнала с помощью \code{sigaction}}

\section{Генерация исключений языка \code{C++}}

\subsection{Вложенные исключения}

\subsection{Использование \code{goto} для выхода из \code{try}}

\section{Выводы}

В процессе выполнения данной работы:

\begin{itemize}
	\item 1
	\item 2
\end{itemize}

\newpage

\section*{Список использованных источников}

\begin{enumerate}
	\item Душутина Е.В. -- Системное программное обеспечение. Практические вопросы разработки системных приложений [Текст] -- 2016.
	\item Таненбаум Э. -- Современные операционные системы [Текст] -- 2015.
	\item Structured Exception Handling / Microsoft Docs [Электронный ресурс]:\\
		{\small\url{https://docs.microsoft.com/en-us/cpp/cpp/structured-exception-handling-c-cpp}}
	\item \_\_ud2 / Microsoft Docs [Электронный ресурс]:\\
		{\small\url{https://docs.microsoft.com/en-us/cpp/intrinsics/ud2}}
	\item Microsoft-specific exception handling mechanisms / Википедия [Электронный ресурс]:\\
		{\small\url{https://en.wikipedia.org/wiki/Microsoft-specific_exception_handling_mechanisms}}
\end{enumerate}

\end{document}
