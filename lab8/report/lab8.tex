\documentclass[a4paper,14pt]{extarticle}

\usepackage[utf8x]{inputenc}
\usepackage[T1]{fontenc}
\usepackage[russian]{babel}
\usepackage{hyperref}
\usepackage{indentfirst}
\usepackage{here}
\usepackage{array}
\usepackage{graphicx}
\usepackage{grffile}
\usepackage{caption}
\usepackage{subcaption}
\usepackage{chngcntr}
\usepackage{amsmath}
\usepackage{amssymb}
\usepackage[left=2cm,right=2cm,top=2cm,bottom=2cm,bindingoffset=0cm]{geometry}
\usepackage{multicol}
\usepackage{multirow}
\usepackage{titlesec}
\usepackage{listings}
\usepackage{listingsutf8}
\usepackage{color}
\usepackage{enumitem}
\usepackage{cmap}
\usepackage{titlesec}

\definecolor{green}{rgb}{0,0.6,0}
\definecolor{gray}{rgb}{0.5,0.5,0.5}
\definecolor{purple}{rgb}{0.58,0,0.82}

\lstdefinelanguage{none}{}

\lstset{
	language={C++},
	inputpath={../},
	backgroundcolor=\color{white},
	commentstyle=\color{green},
	keywordstyle=\color{blue},
	numberstyle=\color{gray}\scriptsize\ttfamily,
	stringstyle=\color{purple},
	basicstyle=\lst@ifdisplaystyle\footnotesize\fi\ttfamily,
	breakatwhitespace=false,
	breaklines=true,
	captionpos=b,
	keepspaces=true,
	numbers=left,
	numbersep=5pt,
	showspaces=false,
	showstringspaces=false,
	showtabs=false,
	tabsize=4,
	frame=single,
	morekeywords={NULL, DWORD, WINAPI, HANDLE, STARTUPINFO, BYTE, LPSTR, SOCKET, WSADATA, TCHAR, LPCTSTR, LPOVERLAPPED, WSABUF, SECURITY_ATTRIBUTES, SECURITY_DESCRIPTOR, TRUE, FALSE, PROCESS_INFORMATION, PIPE_UNLIMITED_INSTANCES, LPVOID, sockaddr_in},
	deletekeywords={error},
	alsoletter={_},
	sensitive=true,
	extendedchars=false,
	columns=fullflexible,
	inputencoding=utf8/cp1251,
	literate=%
		{~}{{\raise.25ex\hbox{$\mathtt{\sim}$}}}{1}%
		{-}{-}{1}
}

\makeatletter
\def\lst@outputspace{{\ }}
\makeatother

\renewcommand{\le}{\ensuremath{\leqslant}}
\renewcommand{\leq}{\ensuremath{\leqslant}}
\renewcommand{\ge}{\ensuremath{\geqslant}}
\renewcommand{\geq}{\ensuremath{\geqslant}}
\renewcommand{\epsilon}{\ensuremath{\varepsilon}}
\renewcommand{\phi}{\ensuremath{\varphi}}
\renewcommand{\thefigure}{\arabic{figure}}
\newcommand{\code}[1]{\lstinline|#1|}
\newcommand{\caret}{\^{}}
\newcommand{\ctrl}[1]{\^{}{#1}}
\newcommand{\listingwithoutput}[1]{
	\lstinputlisting[caption=\code{#1.cpp}]{src/#1/#1.cpp}
	Выполним программу \code{#1.exe}:
	\lstinputlisting[language=none]{logs/#1/#1.txt}
}

\titleformat*{\section}{\large\bfseries}
\titleformat*{\subsection}{\normalsize\bfseries}
\titleformat*{\subsubsection}{\normalsize\bfseries}
\titleformat*{\paragraph}{\normalsize\bfseries}
\titleformat*{\subparagraph}{\normalsize\bfseries}

\titlespacing{\section}{0em}{0.8em}{0.8em}

\counterwithin{figure}{section}
\counterwithin{equation}{section}
\counterwithin{table}{section}
\newcommand{\sign}[1][5cm]{\makebox[#1]{\hrulefill}}
\newcommand{\equipollence}{\quad\Leftrightarrow\quad}
\newcommand{\no}[1]{\overline{#1}}
\graphicspath{{../pics/}}
\captionsetup{justification=centering,margin=1cm}
\def\arraystretch{1.3}
\setlength\parindent{5ex}
\titlelabel{\thetitle.\quad}

\setitemize{topsep=0em, itemsep=0em}
\setenumerate{topsep=0em, itemsep=0em}

\begin{document}

\begin{titlepage}
\begin{center}
	Санкт-Петербургский Политехнический Университет Петра Великого\\[0.3cm]
	Институт компьютерных наук и технологий \\[0.3cm]
	Кафедра компьютерных систем и программных технологий\\[4cm]
	
	\textbf{ОТЧЕТ}\\ 
	\textbf{по лабораторной работе}\\[0.5cm]
	\textbf{<<Процессы и потоки в Windows>>}\\[0.1cm]
	Операционные системы\\[3.0cm]
\end{center}

\begin{flushright}
	\begin{minipage}{0.5\textwidth}
		\textbf{Работу выполнил студент}\\[3mm]
		группа 43501/3 \hfill Дьячков В.В.\\[5mm]
		\textbf{Работу принял преподаватель}\\[5mm]
		\sign[2cm] \hfill к.т.н., доц. Душутина Е.В. \\[5mm]
	\end{minipage}
\end{flushright}

\vfill

\begin{center}
	Санкт-Петербург\\[0.3cm]
	\the\year
\end{center}
\end{titlepage}

\addtocounter{page}{1}

\tableofcontents
\newpage

\section{Цели работы}

Осуществить анализ механизмов обработки аппаратных и программных исключений, доступных в операционной системе Linux.

\section{Программа работы}

\renewcommand{\labelenumii}{\theenumii}
\renewcommand{\theenumii}{\theenumi.\arabic{enumii}.}

\textbf{Порождение и запуск процессов}

\begin{enumerate}
	\item Неименованные каналы (pipe):
		\begin{enumerate}
			\item Создать клиент-серверное приложение, позволяющее набираемые символы в терминальном окне командной строки (сервер) отображать их в окно процесса-потомка (клиент).
			\item Создать эхо-сервер, взаимодействующий с клиентом посредством pipe.
		\end{enumerate}
	\item Именованные каналы (named pipe):
		\begin{enumerate}
			\item Реализовать между одним клиентом и сервером обмен данными, вводимыми с консоли на стороне клиента и возвращаемыми сервером обратно до получения команды exit.
			\item Реализовать между сервером и множеством клиентов обмен данными, вводимыми с консоли на стороне клиента и возвращаемыми сервером обратно до получения команды exit.
			\item Модифицировать приложение из предыдущего примера для сетевого обмена информацией.
		\end{enumerate}
	\item Сокеты (socket):
		\begin{enumerate}
			\item Реализовать программ локального и сетевую обмена с помощью сокетов с использованием потокового протокола с установлением соединения (TCP в стеке TCP/IP).
			\item Модифицировать программу для локального обмена с множеством клиентов и с доступом к общему ресурсу. Провести эксперимент с множеством клиентов при сетевом обмене, представить результаты для виртуальной и реальной сетей.
			\item Проанализировать пример применения сокетов (сетевой обмен «мгновенными» сообщениями).
			\item Привести примеры использования портов завершения. Привести пример приложения с большим количеством клиентов до 1000 (когда порты завершения оправданы), общее количество потоков не более 10.
			\item Реализовать обмен на основе UDP.
		\end{enumerate}
	\item Сигналы (signal):
		\begin{enumerate}
			\item Задать обработчик сигналов завершения для консольного приложения.
			\item Самостоятельно предложить собственную реализацию обработчика сигнала.
		\end{enumerate}
	\item Разделяемая память (file mapping):
		\begin{enumerate}
			\item Создать программу, в которой первый процесс генерирует случайное число и записывает его в буфер, доступный второму процессу, откуда он его и считывает с последующим выводом.
		\end{enumerate}
	\item Почтовые слоты (MailSlot):
		\begin{enumerate}
			\item Предложить собственную реализацию приложения, иллюстрирующую обмен информацией почтовыми слотами. Продемонстрировать возможность локального и удаленного доступа. Выполнить широковещательную передачу данных.
		\end{enumerate}
\end{enumerate}

\section{Используемое окружение}

\begin{itemize}
	\item ОС: Ubuntu
	\item Версия ОС: 19.10
	\item Процессор: Intel® Core™ i7-8550U CPU @ 1.80GHz × 8
	\item ОЗУ: 16 Гб
	\item Компилятор: g++ (Ubuntu 9.2.1-9ubuntu2) 9.2.1 20191008
	\item Отладчик: GNU gdb (Ubuntu 8.3-0ubuntu1) 8.3
\end{itemize}

\section{Используемые утилиты}

\subsection{Отладчик \code{GNU gdb}}

GNU Debugger (\code{GDB}) -- переносимый отладчик проекта GNU, который работает на многих UNIX-подобных системах и умеет производить отладку многих языков программирования, включая \code{C} и \code{C++}. \code{GDB} предлагает обширные средства для слежения и контроля за выполнением компьютерных программ. Пользователь может изменять внутренние переменные программ и вызывать функции независимо от обычного поведения программы.

\noindent Команды запуска отлдачика:
\begin{itemize}
	\item \code{gdb prog} -- отладка программы \code{prog} под управлением отладчика;
	\item \code{gdb prog pid} -- подключение отладчика к выполняющейся программе \code{prog} с ID процесса \code{pid};
	\item \code{gdb -h} -- справка о всех возможных опциях отладчика.
\end{itemize}

\noindent Рассмотрим основные команды интерактивной командной строки gdb:
\begin{itemize}
	\item \code{break func} -- установить точку останова в начало функции \code{func};
	\item \code{break line} -- установить точку останова в строке номер \code{line} исходного файла;
	\item \code{continue} (\code{c}) -- продолжить выполнение программы с места последней точки останова;
	\item \code{step} (\code{s}) -- выполнить следующую строку исходного текста программы (даже если эта строка внутри вызываемой функции) и остановиться;
	\item \code{backtrace} (\code{bt}) -- выдать список вложенных вызовов функций прикладной программы с указанием аргументов и мест вызова;
	\item \code{print expr} (\code{p expr}) -- выдать значение выражения expr;
	\item \code{x[/[n][f][u]] addr} -- выдать содержимое памяти программы по адресу \code{addr}, где \code{n} -- количество объектов, \code{f} -- формат вывода, \code{u} -- размер объекта.
	\item \code{info registers} -- выводит имена и содержимое всех регистров общего назначения;
	\item \code{info variables} -- выдать информацию о всех переменных программы;
	\item \code{set var varname = expr} -- присвоить переменной \code{varname} программы значение выражения \code{expr};
\end{itemize}

\subsection{Утилиты \code{strace} и \code{ltrace}}

Улилита \code{strace} может использоваться в диагностических, учебных или отладочных целях. Эта улилита показывает все системные вызовы программы, которые та отправляет к системе во время выполнения, а также их параметры и результат выполнения. Общий формат запуска утилиты: \code{strace <options> <command> <arguments>}. Общий формат выводимых сообщений: \code{syscall (params) = result}.

Рассмотрим некоторые опции утилиты:
\begin{itemize}
	\item \code{-i} -- выводить указатель на инструкцию во время выполнения системного вызова;
  \item \code{-k} -- выводить стек вызовов для отслеживаемого процесса после каждого системного вызова;
	\item \code{-o} -- выводить всю информацию о системных вызовах не в стандартный поток ошибок, а в файл;
	\item \code{-t} -- выводить время суток для каждого вызова;
	\item \code{-e} -- позволяет отфильтровать только нужные системные вызовы или события.
\end{itemize}

Утилита \code{ltrace} похожа на \code{strace}, но умеет работать не только с системными вызовами, но и с библиотечными вызовами. Опции уилиты сходы с \code{strace}, а запуск с флагом \code{-S} позволяет также перехватить системные вызовы, что позволяет использовать ее аналогично \code{strace}.

\section{Обработка сигналов}

\subsection{Типы сигналов}

Сигналы в системе UNIX  используются для того, чтобы сообщить процессу о том, что возникло какое-либо событие или необходимо обработать исключительное состояние. Сигналы могут возникать синхронно с ошибкой в приложении, например SIGFPE (ошибка вычислений с плавающей запятой) и SIGSEGV (ошибка адресации), но большинство сигналов является асинхронными. Сигналы могут посылаться процессу, если система обнаруживает программное событие, например, когда пользователь дает команду прервать или остановить выполнение, или получен сигнал на завершение от другого процесса. Сигналы могут прийти непосредственно от ядра ОС, когда возникает сбой аппаратных средств ЭВМ. Система определяет набор сигналов, которые могут быть отправлены процессу. В Linux применяется около 30 различных сигналов. При этом каждый сигнал имеет целочисленное значение и приводит к строго определенным действиям.

Изначально были разработаны и использовались простые (ненадежные) сигналы. По своему принципу работы они похожи на канонический механизм обработки аппаратных прерываний в процессоре. Если процесс хочет особым образом обрабатывать некий сигнал, то он сообщает ядру об этом, указывая специальную функцию – обработчик сигнала. При доставке сигнала процессу ядро как можно скорее вызывает обработчик сигнала, прерывая работу процесса. По завершении работы обработчика выполнение процесса продолжается с того места, где он был прерван. Приведем краткое описание базовых сигналов:
\begin{enumerate}[itemsep=-0.2em]
	\item \code{SIGHUP} - Закрытие терминала
	\item \code{SIGINT} -- Сигнал прерывания (\ctrl{C}) с терминала
	\item \code{SIGQUIT} -- Сигнал выхода из терминала (\ctrl{$\backslash$})
	\item \code{SIGILL} -- Несуществующая инструкция
	\item \code{SIGTRAP} -- Ловушка трассировки или ловушка
	\item \code{SIGABRT} -- Сигнал, посланный функцией \code{abort}
	\item \code{SIGBUS} -- Неправильное обращение в физическую память
	\item \code{SIGFPE} -- Ошибочная арифметическая операция
	\item \code{SIGKILL} -- Безусловное завершение
	\item \code{SIGUSR1} -- Пользовательский сигнал №1
	\item \code{SIGSEGV} -- Нарушение при обращении к памяти
	\item \code{SIGUSR2} -- Пользовательский сигнал №2
	\item \code{SIGPIPE} -- Запись в разорванное соединение (пайп, сокет)
	\item \code{SIGALRM} -- Сигнал истечения времени, заданного \code{alarm()}
	\item \code{SIGTERM} -- Сигнал завершения
	\item \code{SIGSTKFLT} -- Переполнение стека сопроцессора
	\item \code{SIGCHLD} -- Потомок остановлен или прекратил выполнение
	\item \code{SIGCONT} -- Продолжить выполнение ранее остановленного процесса
	\item \code{SIGSTOP} -- Остановка выполнения процесса
	\item \code{SIGTSTP} -- Сигнал остановки с терминала (\ctrl{Z})
	\item \code{SIGTTIN} -- ввод с терминала у фонового процесса
	\item \code{SIGTTOU} -- вывод на терминал у фонового процесса
	\item \code{SIGURG} -- На сокете получены срочные данные
	\item \code{SIGXCPU} -- Процесс превысил предел процессорного времени
	\item \code{SIGXFSZ} -- Процесс превысил допустимый размер файла
	\item \code{SIGVTALRM} -- Виртуальный будильник
	\item \code{SIGPROF} -- Закончилось время профилирующего таймера
	\item \code{SIGWINCH} -- Изменение размеров окна
	\item \code{SIGIO} -- I/O теперь возможно
	\item \code{SIGPWR} -- Авария питающего напряжения
	\item \code{SIGSYS} -- Неправильный аргумент процедуры
\end{enumerate}

Начиная с 32 указаны сигналы реального времени. Они были разработаны в процессе поиска решения проблемы появления сигнала во время обработки другого сигнала и имеют несколько ключевых отличий от базовых. Сигналы реального времени помещаются в очередь, если сигнал будет порожден несколько раз, он будет несколько раз получен адресатом. Повторения одного и того же сигнала доставляются в порядке очереди FIFO. Если сигналы в очередь не помещаются, неоднократно порожденный сигнал будет получен лишь один раз. Сигналы реального времени с меньшим номером будут обработаны раньше. Если же одновременно посылаются несколько обычных сигналов, то они сливаются в один.

В данной работе будут исследованы сигналы \code{4 SIGILL} -- несуществующая инструкция и \code{8 SIGFPE} -- ошибочная арифметическая операция.

\subsection{Вспомогательные функции}

Реализуем вспомогательные функции, которые помогут исследовать генерацию и обработку сигналов.

\lstinputlisting[caption=\code{utils.h}]{src/utils/utils.h}

\lstinputlisting[caption=\code{utils.cpp}]{src/utils/utils.cpp}

Рассмотрим реализацию представленных функций:
\begin{itemize}
	\item \code{void generateSigfpe()} -- генерирует сигнал ошибочной арифметической операции \code{SIGFPE}. Для этого выполняется операция деления на ноль.
	\item \code{generateSigill()} -- приводит к генерации сигнала выполнения неправильно сформированной, несуществующей или привилегированной инструкции \code{SIGILL}. Для этого используется функция \code{__buildin_trap()}, которую предоставляет компилятор \code{g++}. Внутри этой функции выполняется несуществующая инструкция для того, чтобы завершить программу ненормально (abnormal exit).
	\item \code{__sighandler_t signalHandler(int sig, __sighandler_t handler)} -- является оберткой над функцией \code{signal} для выставления обработчика \code{handler} для сигнала \code{s}. Сигналы \code{SIGKILL} и \code{SIGSTOP} не могут быть перехвачены или проигнорированы. В качестве обработчика также можно передать \code{SIG_IGN} (игнорирование сигнала) и \code{SIG_DFL} (восстановить обработчик по умолчанию).
	\item \code{__sighandler_t sigactionHandler(int s, __sighandler_t handler)} -- является обёрткой над функцией \code{sigaction} для выставления обработчика \code{handler} для сигнала \code{s}. Функция \code{sigaction} -- это альтернативный способ выставления обработчика для сигнала при помощи одноименной структуры \code{sigaction}. Эта структура имеет следующие важные поля: \code{sa_handler} -- функция-обработчик; \code{sa_mask} -- маска сигналов, которые
	будут блокированы пока выполняется этот обработчик (по умолчанию блокируется и сам полученный сигнал); \code{sa_flags} -- позволяет задать дополнительные действия при обработке сигнала (например, возможность автоматически установить обработчик по умолчанию после первого выполнения обработчика).
	\item \code{void printingHandler(int sig)} -- используется в качестве обработчика в демонстрационных целях. В рамках обработки функция выводит отладочную информацию о полученном сигнале, после чего восстанавливает обработчик по умолчанию.
	\item \code{std::string toString(int sig)} -- вспомогательная функция, которая преобразует номер сигнала в его словесное описание.
\end{itemize}

\subsection{Генерация сигналов}

\subsubsection{\code{SIGFPE}}

Рассмотрим программу генерации сигнала выполнения ошибочной арифметической операции \code{SIGFPE}. Для этого вызовем функцию \code{generateSigfpe}, рассмотренную выше. Стоит отметить, что для генерации произвольного сигнала может быть использована функция \code{void raise(int sig)}.

\listingwithoutput{sigfpe}

Видно, что программа завершилась с кодом ошибки 136, что соответствует ошибочной арифметической операции (в данном случае, делению на 0).

Запустим программу \code{sigfpe} в отладчике \code{gdb}.

\logs{sigfpe/gdb}

Из строки 13 видно, что произошла арифметическая ошибка и программа получила сигнал \code{SIGFPE}, после чего программа была завершена.

\subsubsection{\code{SIGILL}}

Рассмотрим программу генерации сигнала, генерируемого при попытке выполнить неправильно сформированную, несуществующую или привилегированную инструкцию \code{SIGILL}. Для этого будем использовать функцию \code{generateSigill}.

\listingwithoutput{sigill}

Запустим программу \code{sigill} в отладчике \code{gdb}.

\logs{sigill/gdb}

Из строки 13 видно, что была выполнена неправильная инструкция и программа получила сигнала \code{SIGILL}, после чего программа была завершена.

\subsection{Назначение обработчика сигнала с помощью \code{signal}}

\subsubsection{\code{SIGFPE}}

Рассмотрим программу, в которой будет установлен обработчик для сигнала \code{SIGFPE}. Установленный обработчик будет выводить номер полученного сигнала и возвращать обработчик по умолчанию. Для этого вызовем вспомогательную функцию \code{singalHandler}, передав ей номер нужного сигнала и нужный обработчик.

\listingwithoutput{sigfpe_signal}

Запустим программу \code{sigfpe_signal} в отладчике \code{gdb}.

\logs{sigfpe_signal/gdb}

Видно, что инструкция, вызывающая получение сигнала, была выполнена дважды. В первый раз был выполнен установленный обработчик, который вывел информацию о сигнале (с. 20). Во второй раз уже был вызван обработчик по умолчанию, который завершил выполнение программы (с. 29).

\subsubsection{\code{SIGILL}}

Выполним аналогичные действия с сигналом \code{SIGILL}. Воспользуемся той же вспомогательной функцией \code{signalHandler}.

\listingwithoutput{sigill_signal}

Запустим программу \code{sigill_signal} в отладчике \code{gdb}.

\logs{sigill_signal/gdb}

Видно, что функция \code{__buildin_trap} была вызвана дважды. В первый раз сгенерированный сигнал был обработан \code{printingHandler}, а во второй раз был вызван обработчик по умолчанию, завершивший программу.

\subsection{Назначение обработчика сигнала с помощью \code{sigaction}}

\subsubsection{\code{SIGFPE}}

Функция \code{sigaction} может быть использована вместо \code{signal}, чтобы обеспечить надежную передачу сигнала: если при
возникновении сигнала система будет занята обработкой другого сигнала, то возникший сигнал не будет потерян. Его обработка будет отложена до окончания текущего обработчика и применена после этого.

Воспользуемся функцией \code{sigactionHandler} для того, чтобы выставить обработчик сигнала \code{SIGFPE} через функцию \code{sigaction}.

\listingwithoutput{sigfpe_sigaction}

Вывод отладчика \code{gdb} аналогичен предыдущему разделу, поэтому рассмотрим системные вызовы, которые делает разработанная программа, при помощи утилиты \code{strace}.

\logs{sigfpe_sigaction/strace}

Видно, что программа выполняет системный вызов \code{rt_sigaction} (с. 6). Данный вызов происходит при вызове \code{sigaction}. Функция проверяет и устанавливает signal action для переданного сигнала. Во второй раз \code{rt_sigaction} (c. 12) используется для восстановления обработчика по умолчанию. В конце происходит вызов \code{rt_sigreturn} -- возврат из обработчика и отчистка стек фрейма. Кроме того, восстанавливается маска сигналов процесса, происходит переключение стека, восстанавливается контекст процесса (регистры, флаги процессора), после чего процесс продолжает выполнение с того места, где произошло получение сигнала.

\subsubsection{\code{SIGILL}}

Воспользуемся функцией \code{sigactionHandler} для того, чтобы выставить обработчик сигнала \code{SIGILL} через функцию \code{sigaction}.

\listingwithoutput{sigill_sigaction}

Вывод отладчика \code{gdb} аналогичен предыдущему разделу, поэтому рассмотрим системные вызовы, которые делает разработанная программа, при помощи утилиты \code{strace}.

\logs{sigill_sigaction/strace}

Аналогично выводу \code{strace} при генерации сигнала \code{SIGFPE}, программа дважды вызывает \code{rt_sigaction} для того, чтобы выставить обработчик для сигнала, и единожды \code{rt_sigreturn} для возврата из обработчика.

\subsection{Обработка вложенных сигналов}

\subsubsection{Ненадежные сигналы}

Рассмотрим обработку вложенных ненадежных сигналов. Внутри обработчика сигнала будем генерировать другой сигнал.

\listingwithoutput{signal_nested}

Видно, что несмотря на то, что были использованы ненадежные сигналы, вложенный сигнал не был потерян. Рассмотрим последовательность системных вызовов в этой программе.

\logs{signal_nested/strace}

Из вывода \code{strace} видно, что сначала были установлены обработчики (с. 4-9), затем был сгенерирован \code{SIGILL} (с. 10-15). Была выведена информация об обработке (с. 16) и сгенерирован сигнал \code{SIGFPE} (с. 17-21). Затем была выведена информация о сигнале (с. 24) и возвращены результаты обработки сигналов (с. 26-27).

\subsubsection{Надежные сигналы}

Проделаем аналогичные действия с надежными сигналами.

\listingwithoutput{sigaction_nested}

Вывод программы и утилиты \code{strace} полностью совпали с предыдущем разделом. В данном случае, выполнение с надежными и ненадежными сигналами оказалось одинаковым.

\section{Исключения языка \code{C++}}

\subsection{Вспомогательные функции}

Реализуем вспомогательные функции, которые помогут исследовать генерацию и обработку исключений в стиле \code{C++}.

\lstinputlisting[caption=\code{exceptions.h}]{src/exceptions/exceptions.h}

\lstinputlisting[caption=\code{exceptions.cpp}]{src/exceptions/exceptions.cpp}

Создан класс \code{Exception}, который хранит внутри себя информацию о номере сигнала и метке (понадобится при изучении вложенных исключений). Кроме того, реализована функция \code{throwException} для бросания исключения при помощи оператора \code{throw}.

\subsection{Генерация исключений}

С точки зрения программы все равно, какой номер сигнала будет передан в конструктор класса \code{Exception}, поэтому будем исследовать только сигнал \code{SIGILL}.

Сгенерируем два исключения, при этом в первый раз обернем вызов \code{throwException} в конструкцию \code{try-catch}.

\listingwithoutput{exception_try_catch}

Видно, что в первом случае исключение было перехвачено блоком \code{catch} и выведена информация об исключении. Второе исключение не было перехвачено, из-за чего в программе была вызвана функция \code{abort}, которая послала программе сигнал \code{SIGABRT}, после чего программа была завершена с кодом 134 (соответствует получению сигнала \code{SIGABRT}).

Проверим, что при необработанном исключении процесс получает сигнал \code{SIGABRT}. Для этого добавим обработчик для этого сигнала.

\listingwithoutput{exception_signal}

Из вывода программы видно, что процесс получил сигнал \code{SIGABRT} из-за необработанного исключения языка \code{C++}.

Исследуем программу \code{exception_signal} при помощи утилиты \code{strace}.

\logs{exception_signal/strace}

Из вывода видно, что сначала был установлен обработчик для сигнала \code{SIGABRT} (с. 6). При необработанном исключении была дважды вызвана функция \code{rt_sigprocmask} для изменения маски сигналов процесса. Из маски блокируемых сигналов был удален сигнал \code{SIGABRT} (с. 11) и добавлен сигнал реального времени (с. 12). Затем при помощи функции \code{tgkill} программе был отправлен сигнал \code{SIGABRT} (с. 13-15). Установленный обработчик вывел информацию о сигнале (с. 18) и восстановил обработчик по умолчанию (с. 20). Затем программа получила сигнал \code{SIGABRT} вновь (с. 24-26) и в этот раз уже была завершена (с. 29).

Аналогичный вывод можно получить при использовании утилиты \code{ltrace} с ключом \code{-S} (выводить системные вызовы).

\logs{exception_signal/ltrace}

Убедимся в том, что при обработке исключения происходит раскрутка стека. Для этого напишем простую программу, состоящую только из генерации исключения. Запустим ее в \code{gdb} и выполним команду \code{rbreak Unwind}, чтобы добавить точки останова в те функции, в названии которых есть слово unwind (раскрутка).

\lstinputlisting[caption=\code{exception_unwind.cpp}]{src/exception_unwind/exception_unwind.cpp}

\logs{exception_unwind/gdb}

Видно, что было добавлено 42 точки останова и во время обработки исключения отладчик останавливался в некоторых точках. Таким образом, был зафиксирован процесс раскрутки стека во время обработки исключения.

\subsection{Вложенные исключения}

Рассмотрим обработку вложенных исключений. Для этого создадим несколько уровней вложенности конструкций \code{try-catch}.

\listingwithoutput{exception_nested}

Видно, что первый обработчик (программа, с. 9) был проигнорирован, т.к. тип исключения (\code{Exception}) не совпал с ожидаемым (\code{int}). Программа продолжила поиск подходящего обработчика и он был найден (с. 13), о чем свидетельствует вывод в консоль. Внутри блока \code{catch} было брошено еще одно исключение (с. 15), которое было также обработано (с. 18).

\subsection{Использование \code{goto} для выхода из блока \code{try}}

Рассмотрим выход из охраняемого блока \code{try} при помощи оператора безусловного перехода \code{goto}.

\listingwithoutput{exception_goto}

Видно что вывелись только строки перед \code{goto} и после метки \code{lbl}. При помощи \code{gdb} проверим, что стек не был раскручен.

\logs{exception_goto/gdb}

Из вывода видно, что было установлено 42 точки останова на функциях, связанных с раскруткой стека, но ни одна из этих точек не сработала. Следовательно, при использовании \code{goto} раскрутки стека не происходит.

\section{Работа с системным журналом}

Журналирование является основным источником информации о работе системы и ее ошибках. Большинство лог файлов в Linux содержится в директории \code{/var/log}. Системный журнал \code{/var/log/syslog} содержит глобальный системный журнал, в который пишутся сообщения с момента запуска системы, от ядра Linux, различных служб, обнаруженных устройствах, сетевых интерфейсов и клиентских приложений.

Для работы с системным журналом используется три основные функции:
\begin{itemize}
	\item \code{void openlog(const char *ident, int option, int facility)} -- устанавливает связь с программой, ведущей системный журнал. Строка \code{ident} добавляется к каждому сообщению и обычно представляет собой название программы. Через \code{option} передаются флаги для управления работой \code{openlog} и последующих вызовов \code{syslog}.
	\item \code{void syslog(int priority, const char *format, ...)} -- создает сообщение, которое передается в журнал.
	\item \code{void closelog()} -- закрывает дескриптор, используемый для записи данных в журнал.
\end{itemize}

Реализуем обработчик для сигнала, который будет логировать информацию об обработке события в системный журнал используя перечисленные функции.

\listingwithoutput{logsys}

Из результатов видно, что обработчик успешно записал сообщение об обработке сигнала в журнал \code{/var/log/syslog}.

\section{Обращение к порту}

Перед тем, как вы получите доступ к порту, программе необходимо получить права на это. Это выполняется, при помощи функции \code{ioperm(from, num, turn_on)} (определенной в \code{sys/io.h} и находящейся в ядре), где \code{from} -- это первый порт, а \code{num} это количество подряд идущих портов, которым нужно дать доступ. Последний аргумент -- это двоичное значение, определяющее, дать ли доступ к портам (\code{1}) или запретить его (\code{0}).

Для вызова функции \code{ioperm} необходимо иметь права \code{root}. Таким образом, программа должна быть запущена от пользователя \code{root} или должен быть установлен на файл флаг \code{setuid}. Функция \code{ioperm} может дать доступ только к портам с \code{0x000} по \code{0x3ff}.

Для чтения одного байта из порта нужно использовать функцию \code{inpb}, а для вывода \code{outb}. Рассмотрим программу, в которой запрашивается доступ к порту \code{0x378}, затем в него пишется значение 0. Наконец, из этого порта считывается один байт и выводится на консоль.

\listingwithoutput{ioport}

Видно, что без прав суперпользователя не удалось получить доступ к порту, однако при помощи \code{sudo} доступ был получен. После этого было записано значение \code{0} и считано значение \code{255}. 

Рассмотрим, какие системные вызовы делаются в этой программе.

\logs{ioport/strace}

Видно, что делается только один системный вызов, связанный с вводом-выводом из порта -- запрос на доступ к порту через функцию \code{ioport}. 

\section{Выводы}

В процессе выполнения данной работы:

\begin{itemize}
	\item изучены отладчик \code{gdb} и утилиты \code{strace} и \code{ltrace};
	\item проведен анализ обработки надежных и нанадежных сигналов в Linux: разработаны функции генерации сигналов, рассмотрена обработка вложенных сигналов;
	\item изучена обработка исключений языка \code{C++}: генерация исключений и обработка вложенных исключений;
	\item рассмотрена работа с системным журналом \code{syslog};
	\item проведен опыт с обращением к порту.
\end{itemize}

\newpage

\section*{Список использованных источников}

\begin{enumerate}
	\item Душутина Е.В. -- Системное программное обеспечение. Практические вопросы разработки системных приложений [Текст] -- 2016.
	\item Таненбаум Э. -- Современные операционные системы [Текст] -- 2015.
	\item Linux processes and signals / BogoToBogo [Электронный ресурс]:\\
		{\small\url{https://www.bogotobogo.com/Linux/linux_process_and_signals.php}}
	\item Other Built-in Functions Provided by GCC / GCC Docs [Электронный ресурс]:
		{\small\url{https://gcc.gnu.org/onlinedocs/gcc/Other-Builtins.html}}
	\item rt\_sigaction(2) - Linux man page [Электронный ресурс]:\\
		{\small\url{https://linux.die.net/man/2/rt_sigaction}}
	\item Краткое описание отладчика GDB [Электронный ресурс]:\\
		{\small\url{http://www.igce.comcor.ru/gdb_briefly.html}}
	\item Команда strace в Linux [Электронный ресурс]:\\
		{\small\url{https://losst.ru/komanda-strace-v-linux}}
	\item Понятие о сигналах -- OpenNET [Электронный ресурс]:\\
		{\small\url{https://www.opennet.ru/docs/RUS/linux_parallel/node10.html}}
	\item Linux I/O port programming -- OpenNET [Электронный ресурс]:\\
		{\small\url{https://www.opennet.ru/docs/HOWTO-RU/mini/IO-Port-Programming.html}}
\end{enumerate}

\end{document}
