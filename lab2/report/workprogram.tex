\renewcommand{\labelenumii}{\theenumii}
\renewcommand{\theenumii}{\theenumi.\arabic{enumii}.}

\begin{enumerate}
	\item Ознакомиться с типами файлов исследуемой ФС. Применяя утилиту \code{ls}, отфильтровать по одному примеру каждого типа файла используемой вами ФС. Комбинируя различные ключи утилиты рекурсивно просканировать все дерево, анализируя крайнюю левую позицию выходной информации полученной посредством \code{ls –l}. Результат записать в выходной файл с указанием полного пути каждого примера. Выполнить задание сначала в консоли построчно, выбирая необходимые сочетания ключей (в командной строке), а затем оформить как скрипт с задаваемым в командной строке именем файла как параметр.
	\item Получить все жесткие ссылки на заданный файл, находящиеся в разных каталогах пользовательского пространства (разными способами, не применяя утилиты \code{file} и \code{find}). Использовать конвейеризацию и фильтрацию. Оформить в виде скрипта.
	\item Проанализировать все возможные способы формирования символьных ссылок (\code{ln}, \code{link}, \code{cp} и т.д.), продемонстрировать их экспериментально. Предложить скрипт, подсчитывающий и перечисляющий все полноименные символьные ссылки на файл, размещаемые в разных местах файлового дерева.
	\item Получить все символьные ссылки на заданный в качестве входного параметра файл, не используя \code{file} (разными способами, не применяя утилиту \code{file}).
	\item Изучить утилиту \code{find}, используя ее ключи получить расширенную информацию о всех типах файлов. Создать примеры вложенных команд.
	\item Проанализировать содержимое заголовка файла, а также файла-каталога с помощью утилит \code{od} и \code{*dump}. Если доступ к файлу-каталогу возможен (для отдельных модификаций POSIX-совместимых ОС), проанализировать изменение его содержимого при различных операциях над элементами, входящими в его состав (файлами и подкаталогами).
	\item Определить максимальное количество записей в каталоге. Изменить размер каталога, варьируя количество записей (для этого создать программу, порождающую новые файлы и каталоги, а затем удаляющую их, предусмотрев промежуточный и конечный вывод информации о размере подопытного каталога).
	\item Ознакомиться с содержимым \code{/etc/passwd}, \code{/etc/shadow}, с утилитой \code{/usr/bin/passwd}, проанализировать права доступа к этим файлам.
	\item Исследовать права владения и доступа, а также их сочетаемость
		\begin{enumerate}
			\item Привести примеры применения утилит \code{chmod}, \code{chown} к специально созданному для этих целей отдельному каталогу с файлами.
			\item Расширить права исполнения экспериментального файла с помощью флага SUID.
			\item Экспериментально установить, как формируются итоговые права на использование файла, если права пользователя и группы, в которую он входит, различны.
			\item Сопоставить возможности исполнения наиболее часто используемых операций, варьируя правами доступа к файлу и каталогу.
		\end{enumerate}
	\item Разработать «программу-шлюз» для доступа к файлу другого пользователя при отсутствии прав на чтение информации из этого файла. Провести эксперименты для случаев, когда пользователи принадлежат одной и разным группам. Сравнить результаты. Для выполнения задания применить подход, аналогичный для обеспечения функционирования утилиты \code{/usr/bin/passwd} (манипуляции с правами доступа, флагом SUID, а также размещением файлов).
	\item Применяя утилиту df и аналогичные ей по функциональности утилиты, а также информационные файлы типа \code{fstab}, получить информацию о файловых системах, возможных для монтирования, а также установленных на компьютере реально.
		\begin{enumerate}
			\item Привести информацию об исследованных утилитах и информационных файлах с анализом их содержимого и форматов.
			\item Привести образ диска с точки зрения состава и размещения всех ФС на испытуемом компьютере, а также образ полного дерева ФС, включая присоединенные ФС съемных и несъемных носителей. Проанализировать и указать формат таблицы монтирования.
			\item Привести «максимально возможное» дерево ФС, проанализировать, где это указывается.
		\end{enumerate}
	\item Проанализировать и пояснить принцип работы утилиты file.
		\begin{enumerate}
			\item Привести алгоритм её функционирования на основе информационной базы, размещение и полное имя которой указывается в описании утилиты в технической документации ОС, а также содержимого заголовка файла, к которому применяется утилита. Определить, где находятся магические числа и иные характеристики, идентифицирующие тип файла, применительно к исполняемым файлам, а также файлам других типов.
			\item Утилиту \code{file} выполнить с разными ключами.
			\item Привести экспериментальную попытку с добавлением в базу собственного типа файла и его дальнейшей идентификацией. Описать эксперимент и привести последовательность действий для расширения функциональности утилиты \code{file} и возможности встраивания дополнительного типа файла в ФС (согласовать содержимое информационной базы и заголовка файла нового типа).
		\end{enumerate}

\end{enumerate}