\documentclass[a4paper,14pt]{extarticle}

\usepackage[utf8x]{inputenc}
\usepackage[T1]{fontenc}
\usepackage[russian]{babel}
\usepackage{hyperref}
\usepackage{indentfirst}
\usepackage{here}
\usepackage{array}
\usepackage{graphicx}
\usepackage{grffile}
\usepackage{caption}
\usepackage{subcaption}
\usepackage{chngcntr}
\usepackage{amsmath}
\usepackage{amssymb}
\usepackage[left=2cm,right=2cm,top=2cm,bottom=2cm,bindingoffset=0cm]{geometry}
\usepackage{multicol}
\usepackage{multirow}
\usepackage{titlesec}
\usepackage{listings}
\usepackage{listingsutf8}
\usepackage{color}
\usepackage{enumitem}
\usepackage{cmap}
\usepackage{titlesec}

\definecolor{green}{rgb}{0,0.6,0}
\definecolor{gray}{rgb}{0.5,0.5,0.5}
\definecolor{purple}{rgb}{0.58,0,0.82}

\lstdefinelanguage{none}{}

\lstset{
	language={C++},
	inputpath={../},
	backgroundcolor=\color{white},
	commentstyle=\color{green},
	keywordstyle=\color{blue},
	numberstyle=\color{gray}\scriptsize\ttfamily,
	stringstyle=\color{purple},
	basicstyle=\lst@ifdisplaystyle\footnotesize\fi\ttfamily,
	breakatwhitespace=false,
	breaklines=true,
	captionpos=b,
	keepspaces=true,
	numbers=left,
	numbersep=5pt,
	showspaces=false,
	showstringspaces=false,
	showtabs=false,
	tabsize=4,
	frame=single,
	morekeywords={NULL, DWORD, WINAPI, HANDLE, STARTUPINFO, BYTE, LPSTR, SOCKET, WSADATA, TCHAR, LPCTSTR, LPOVERLAPPED, WSABUF, SECURITY_ATTRIBUTES, SECURITY_DESCRIPTOR, TRUE, FALSE, PROCESS_INFORMATION, PIPE_UNLIMITED_INSTANCES, LPVOID, sockaddr_in},
	deletekeywords={error},
	alsoletter={_},
	sensitive=true,
	extendedchars=false,
	columns=fullflexible,
	inputencoding=utf8/cp1251,
	literate=%
		{~}{{\raise.25ex\hbox{$\mathtt{\sim}$}}}{1}%
		{-}{-}{1}
}

\makeatletter
\def\lst@outputspace{{\ }}
\makeatother

\renewcommand{\le}{\ensuremath{\leqslant}}
\renewcommand{\leq}{\ensuremath{\leqslant}}
\renewcommand{\ge}{\ensuremath{\geqslant}}
\renewcommand{\geq}{\ensuremath{\geqslant}}
\renewcommand{\epsilon}{\ensuremath{\varepsilon}}
\renewcommand{\phi}{\ensuremath{\varphi}}
\renewcommand{\thefigure}{\arabic{figure}}
\newcommand{\code}[1]{\lstinline|#1|}
\newcommand{\caret}{\^{}}
\newcommand{\ctrl}[1]{\^{}{#1}}
\newcommand{\listingwithoutput}[1]{
	\lstinputlisting[caption=\code{#1.cpp}]{src/#1/#1.cpp}
	Выполним программу \code{#1.exe}:
	\lstinputlisting[language=none]{logs/#1/#1.txt}
}

\titleformat*{\section}{\large\bfseries}
\titleformat*{\subsection}{\normalsize\bfseries}
\titleformat*{\subsubsection}{\normalsize\bfseries}
\titleformat*{\paragraph}{\normalsize\bfseries}
\titleformat*{\subparagraph}{\normalsize\bfseries}

\titlespacing{\section}{0em}{0.8em}{0.8em}

\counterwithin{figure}{section}
\counterwithin{equation}{section}
\counterwithin{table}{section}
\newcommand{\sign}[1][5cm]{\makebox[#1]{\hrulefill}}
\newcommand{\equipollence}{\quad\Leftrightarrow\quad}
\newcommand{\no}[1]{\overline{#1}}
\graphicspath{{../pics/}}
\captionsetup{justification=centering,margin=1cm}
\def\arraystretch{1.3}
\setlength\parindent{5ex}
\titlelabel{\thetitle.\quad}

\setitemize{topsep=0em, itemsep=0em}
\setenumerate{topsep=0em, itemsep=0em}

\begin{document}

\begin{titlepage}
\begin{center}
	Санкт-Петербургский Политехнический Университет Петра Великого\\[0.3cm]
	Институт компьютерных наук и технологий \\[0.3cm]
	Кафедра компьютерных систем и программных технологий\\[4cm]
	
	\textbf{ОТЧЕТ}\\ 
	\textbf{по лабораторной работе}\\[0.5cm]
	\textbf{<<Процессы и потоки в Windows>>}\\[0.1cm]
	Операционные системы\\[3.0cm]
\end{center}

\begin{flushright}
	\begin{minipage}{0.5\textwidth}
		\textbf{Работу выполнил студент}\\[3mm]
		группа 43501/3 \hfill Дьячков В.В.\\[5mm]
		\textbf{Работу принял преподаватель}\\[5mm]
		\sign[2cm] \hfill к.т.н., доц. Душутина Е.В. \\[5mm]
	\end{minipage}
\end{flushright}

\vfill

\begin{center}
	Санкт-Петербург\\[0.3cm]
	\the\year
\end{center}
\end{titlepage}

\addtocounter{page}{1}

\tableofcontents
\newpage

\section{Цели работы}

\begin{enumerate}
	\item Изучение основных команд пользовательского интерфейса;
	\item Изучение цикла подготовки и исполнения программ;
	\item Изучение команд и утилит обработки текстов.
\end{enumerate}

\section{Утилиты информации о системе и пользователе}

\subsection{\code{date}}

Вывод текущей даты:
\lstinputlisting[firstline=1, lastline=2]{basic/1}

\subsection{\code{who}}

Пользователи, находящиеся в системе:
\lstinputlisting[firstline=3, lastline=5]{basic/1}

Имя текущего пользователя:
\lstinputlisting[firstline=6, lastline=7]{basic/1}

\subsection{\code{tty}}

Полное имя файла-терминала, подключенного к стандартному вводу:
\lstinputlisting[firstline=8, lastline=10]{basic/1}

\subsection{\code{logname}}

Входное имя пользователя:
\lstinputlisting[firstline=10, lastline=11]{basic/1}

\subsection{\code{uname}}

Имя ядра (kernel):
\lstinputlisting[firstline=12, lastline=13]{basic/1}

\subsection{\code{sleep}}

Команда \code{sleep} приостанавливает выполнение на число секунд, указанное первым аргументом. Например, первый вызов привел к 5-секундной задержке, а второй вызов задерживает на 10000 секунд. Чтобы продолжить выполнение был послан сигнал \code{SIGINT} при помощи сочетания клавиш \code{<\caret C>}, сразу после этого появилось приглашение к вводу.
\lstinputlisting{basic/2}

\subsection{\code{man}}

Команда \code{man} используется для получения справочной информации  о различных программах, утилитах или функциях. Дополнительным аргументом команды является номер раздела (по умолчанию поиск проводится во всех разделах):
\begin{enumerate}
	\item Выполняемые программы или \code{shell} команды;
	\item Системные вызовы (функции, предоставляемые ядром);
	\item Библиотечные вызовы (функции внутри библиотек);
	\item Специальные файлы (обычно \code{/dev});
	\item Конвенции и форматы файлов (например, \code{/etc/passwd});
	\item Игры;
	\item Разное;
	\item Команды системного администрирования (обычно для \code{root});
	\item Подпрограммы ядра (не стандартизован).
\end{enumerate}
\lstinputlisting[deletekeywords={date, who, tty}]{basic/3}

\section{Утилиты работы с файлами (каталогами) и процессами}

\subsection{\code{ls}}

\lstinputlisting[inputencoding=cp1251]{basic/4a_ru}

\subsection{\code{cat}}

\subsection{\code{mv}}

\subsection{\code{cp}}

\subsection{\code{rm}}

\subsection{\code{pwd}}

\subsection{\code{cd}}

\subsection{\code{mkdir}}

\subsection{\code{ps}}

\newpage

\section{Утилиты фильтрации и обработки текста}

\subsection{\code{grep}}

\subsection{\code{cut}}

\subsection{\code{tr}}

\subsection{\code{sort}}

\subsection{\code{uniq}}

\subsection{\code{cmp}}

\subsection{\code{diff}}

\section{Конвейеры}

\subsection{\code{|}}

\subsection{\code{||}}

\subsection{\code{\&\&}}

\section{Цикл обработки программ}

\subsection{Компиляторы и интерпретаторы в системе}

\subsection{\code{cc}}

%\section{Автоматизированное логирование}

\section{Выводы}

\end{document}
