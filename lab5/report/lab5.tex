\documentclass[a4paper,14pt]{extarticle}

\usepackage[utf8x]{inputenc}
\usepackage[T1]{fontenc}
\usepackage[russian]{babel}
\usepackage{hyperref}
\usepackage{indentfirst}
\usepackage{here}
\usepackage{array}
\usepackage{graphicx}
\usepackage{grffile}
\usepackage{caption}
\usepackage{subcaption}
\usepackage{chngcntr}
\usepackage{amsmath}
\usepackage{amssymb}
\usepackage[left=2cm,right=2cm,top=2cm,bottom=2cm,bindingoffset=0cm]{geometry}
\usepackage{multicol}
\usepackage{multirow}
\usepackage{titlesec}
\usepackage{listings}
\usepackage{listingsutf8}
\usepackage{color}
\usepackage{enumitem}
\usepackage{cmap}
\usepackage{titlesec}

\definecolor{green}{rgb}{0,0.6,0}
\definecolor{gray}{rgb}{0.5,0.5,0.5}
\definecolor{purple}{rgb}{0.58,0,0.82}

\lstdefinelanguage{none}{}

\lstset{
	language={C++},
	inputpath={../},
	backgroundcolor=\color{white},
	commentstyle=\color{green},
	keywordstyle=\color{blue},
	numberstyle=\color{gray}\scriptsize\ttfamily,
	stringstyle=\color{purple},
	basicstyle=\lst@ifdisplaystyle\footnotesize\fi\ttfamily,
	breakatwhitespace=false,
	breaklines=true,
	captionpos=b,
	keepspaces=true,
	numbers=left,
	numbersep=5pt,
	showspaces=false,
	showstringspaces=false,
	showtabs=false,
	tabsize=4,
	frame=single,
	morekeywords={NULL, DWORD, WINAPI, HANDLE, STARTUPINFO, BYTE, LPSTR, SOCKET, WSADATA, TCHAR, LPCTSTR, LPOVERLAPPED, WSABUF, SECURITY_ATTRIBUTES, SECURITY_DESCRIPTOR, TRUE, FALSE, PROCESS_INFORMATION, PIPE_UNLIMITED_INSTANCES, LPVOID, sockaddr_in},
	deletekeywords={error},
	alsoletter={_},
	sensitive=true,
	extendedchars=false,
	columns=fullflexible,
	inputencoding=utf8/cp1251,
	literate=%
		{~}{{\raise.25ex\hbox{$\mathtt{\sim}$}}}{1}%
		{-}{-}{1}
}

\makeatletter
\def\lst@outputspace{{\ }}
\makeatother

\renewcommand{\le}{\ensuremath{\leqslant}}
\renewcommand{\leq}{\ensuremath{\leqslant}}
\renewcommand{\ge}{\ensuremath{\geqslant}}
\renewcommand{\geq}{\ensuremath{\geqslant}}
\renewcommand{\epsilon}{\ensuremath{\varepsilon}}
\renewcommand{\phi}{\ensuremath{\varphi}}
\renewcommand{\thefigure}{\arabic{figure}}
\newcommand{\code}[1]{\lstinline|#1|}
\newcommand{\caret}{\^{}}
\newcommand{\ctrl}[1]{\^{}{#1}}
\newcommand{\listingwithoutput}[1]{
	\lstinputlisting[caption=\code{#1.cpp}]{src/#1/#1.cpp}
	Выполним программу \code{#1.exe}:
	\lstinputlisting[language=none]{logs/#1/#1.txt}
}

\titleformat*{\section}{\large\bfseries}
\titleformat*{\subsection}{\normalsize\bfseries}
\titleformat*{\subsubsection}{\normalsize\bfseries}
\titleformat*{\paragraph}{\normalsize\bfseries}
\titleformat*{\subparagraph}{\normalsize\bfseries}

\titlespacing{\section}{0em}{0.8em}{0.8em}

\counterwithin{figure}{section}
\counterwithin{equation}{section}
\counterwithin{table}{section}
\newcommand{\sign}[1][5cm]{\makebox[#1]{\hrulefill}}
\newcommand{\equipollence}{\quad\Leftrightarrow\quad}
\newcommand{\no}[1]{\overline{#1}}
\graphicspath{{../pics/}}
\captionsetup{justification=centering,margin=1cm}
\def\arraystretch{1.3}
\setlength\parindent{5ex}
\titlelabel{\thetitle.\quad}

\setitemize{topsep=0em, itemsep=0em}
\setenumerate{topsep=0em, itemsep=0em}

\begin{document}

\begin{titlepage}
\begin{center}
	Санкт-Петербургский Политехнический Университет Петра Великого\\[0.3cm]
	Институт компьютерных наук и технологий \\[0.3cm]
	Кафедра компьютерных систем и программных технологий\\[4cm]
	
	\textbf{ОТЧЕТ}\\ 
	\textbf{по лабораторной работе}\\[0.5cm]
	\textbf{<<Процессы и потоки в Windows>>}\\[0.1cm]
	Операционные системы\\[3.0cm]
\end{center}

\begin{flushright}
	\begin{minipage}{0.5\textwidth}
		\textbf{Работу выполнил студент}\\[3mm]
		группа 43501/3 \hfill Дьячков В.В.\\[5mm]
		\textbf{Работу принял преподаватель}\\[5mm]
		\sign[2cm] \hfill к.т.н., доц. Душутина Е.В. \\[5mm]
	\end{minipage}
\end{flushright}

\vfill

\begin{center}
	Санкт-Петербург\\[0.3cm]
	\the\year
\end{center}
\end{titlepage}

\addtocounter{page}{1}

\tableofcontents
\newpage

\section{Цели работы}

Изучить средства межпроцессного взаимодействия в ОС семейства UNIX на примере Linux:

\begin{itemize}
	\item сигналы;
	\item анонимные (неименованные) каналы;
	\item именованные каналы;
	\item очереди сообщений;
	\item семафоры;
	\item разделяемая память;
	\item сокеты.
\end{itemize}

\section{Программа работы}

\renewcommand{\labelenumii}{\theenumii}
\renewcommand{\theenumii}{\theenumi.\arabic{enumii}.}

\textbf{Порождение и запуск процессов}

\begin{enumerate}
	\item Неименованные каналы (pipe):
		\begin{enumerate}
			\item Создать клиент-серверное приложение, позволяющее набираемые символы в терминальном окне командной строки (сервер) отображать их в окно процесса-потомка (клиент).
			\item Создать эхо-сервер, взаимодействующий с клиентом посредством pipe.
		\end{enumerate}
	\item Именованные каналы (named pipe):
		\begin{enumerate}
			\item Реализовать между одним клиентом и сервером обмен данными, вводимыми с консоли на стороне клиента и возвращаемыми сервером обратно до получения команды exit.
			\item Реализовать между сервером и множеством клиентов обмен данными, вводимыми с консоли на стороне клиента и возвращаемыми сервером обратно до получения команды exit.
			\item Модифицировать приложение из предыдущего примера для сетевого обмена информацией.
		\end{enumerate}
	\item Сокеты (socket):
		\begin{enumerate}
			\item Реализовать программ локального и сетевую обмена с помощью сокетов с использованием потокового протокола с установлением соединения (TCP в стеке TCP/IP).
			\item Модифицировать программу для локального обмена с множеством клиентов и с доступом к общему ресурсу. Провести эксперимент с множеством клиентов при сетевом обмене, представить результаты для виртуальной и реальной сетей.
			\item Проанализировать пример применения сокетов (сетевой обмен «мгновенными» сообщениями).
			\item Привести примеры использования портов завершения. Привести пример приложения с большим количеством клиентов до 1000 (когда порты завершения оправданы), общее количество потоков не более 10.
			\item Реализовать обмен на основе UDP.
		\end{enumerate}
	\item Сигналы (signal):
		\begin{enumerate}
			\item Задать обработчик сигналов завершения для консольного приложения.
			\item Самостоятельно предложить собственную реализацию обработчика сигнала.
		\end{enumerate}
	\item Разделяемая память (file mapping):
		\begin{enumerate}
			\item Создать программу, в которой первый процесс генерирует случайное число и записывает его в буфер, доступный второму процессу, откуда он его и считывает с последующим выводом.
		\end{enumerate}
	\item Почтовые слоты (MailSlot):
		\begin{enumerate}
			\item Предложить собственную реализацию приложения, иллюстрирующую обмен информацией почтовыми слотами. Продемонстрировать возможность локального и удаленного доступа. Выполнить широковещательную передачу данных.
		\end{enumerate}
\end{enumerate}

\newpage

\section{Используемая операционная система}

\begin{itemize}
	\item Дистрибутив: Ubuntu 18.04.1 LTS
	\item Процессоры: Intel® Core™ i7-4800MQ CPU @ 2.70GHz × 8
	\item Версия ядра: 4.15.0-34-generic
\end{itemize}

\section{Сигналы}

\subsection{Ненадежные сигналы}

Создать программу, позволяющую изменить диспозицию сигналов, а именно, установить:

\begin{itemize}
	\item обработчик пользовательских сигналов \code{SIGUSR1} и \code{SIGUSR2};
	\item реакцию по умолчанию на сигнал \code{SIGINT};
	\item игнорирование сигнала \code{SIGCHLD};
\end{itemize}

Повторить эксперимент для других сигналов, для процессов, порождаемых в разных в разных файлов, для потоков одного процесса, для потоков разных процессов.

\lstinputlisting[caption=\code{sigexam.c}]{src/sigexam/sigexam.c}

\subsection{Надежные сигналы}

Создать программу, позволяющую продемонстрировать возможность отложенной обработки (временного блокирования) сигнала (например, \code{SIGINT}).

Изменить обработчик сигнала так, чтобы из него производилась отправка другого сигнала.

\subsection{Сигналы реального времени}

Проведите эксперимент, позволяющий определить возможность организации очереди для различных типов сигналов, обычных и реального времени, (более двух сигналов, для этого увеличьте «вложенность» вызовов обработчиков). 

Экспериментально подтвердите, что обработка равноприоритетных сигналов реального времени происходит в порядке FIFO. 

Опытным путем подтвердите наличие приоритетов сигналов реального времени.

\section{Каналы}

\subsection{Неименованные каналы}

Организуем программу \code{pipe.c} так, чтобы процесс-родитель создавал неименованный канал, создавал потомка, закрывал канал на запись и записывал в произвольный текстовый файл считываемую из канала информацию. В функции процесса-потомка будет входить считывание данных из файла и запись их в канал.

\subsection{Именованные каналы}

Создать клиент-серверное приложение, демонстрирующее дуплексную (двунаправленную) передачу информации двумя однонаправленными именованными каналами между клиентом и сервером.

\section{Очереди сообщений}

Создать клиент-серверное приложение, демонстрирующее передачу информации между процессами посредством очередей сообщений.

К условиям предыдущей задачи добавляется наличие не единичного буфера, а буфера некоторого размера.

\section{Семафоры и разделяемая память}

Есть один процесс, выполняющий запись в разделяемую память и один процесс, выполняющий чтение из нее. Под чтением понимается извлечение данных из памяти. Программа должна обеспечить невозможность повторного чтения одних и тех же данных и невозможность перезаписи данных, т.е. новой записи, до тех пор, пока читатель не прочитает предыдущую.

К условиям предыдущей задачи добавляется условие корректной работы нескольких читателей и нескольких писателей одновременно. Как и в предыдущем варианте под чтением понимается извлечение данных из памяти, т. е. одну порцию данных может прочитать только один читатель.

К условиям предыдущей задачи добавляется наличие не единичного буфера, а буфера некоторого размера. Тип буфера (очередь, стек, кольцевой буфер) не имеет значения.

\section{Сокеты}

\subsection{TCP}

Сервер прослушивает заданный порт, при приходе нового соединения, создается новый поток для его обработки. Работа с клиентом организована как бесконечный цикл, в котором выполняется прием сообщения от клиента, вывод его на экран и пересылка обратно клиенту.

Модифицировать, если необходимо, предложенное приложение и реализовать обмен сервера с множеством клиентов. Количество клиентов: 10, 100, 1000.

\subsection{UDP}

Выполнить аналогичное взаимодействие на основе UDP.

\section{Выводы}

В процессе выполнения данной работы:

\begin{itemize}
	\item изучены различные типы сигналов: ненадежные, надежные и сигналы реального времени;
	\item рассмотрены неименованные и именованные каналы, реализующие запись по принципу FIFO;
	\item изучены очереди сообщений для асинхронного обмена данными между процессами;
	\item проанализированы варианты использования семафоров для доступа к разделяемой памяти;
	\item рассмотрены TCP и UDP сокеты как средство межпроцессного и межсетевого взаимодействия.
\end{itemize}

\section*{Список использованных источников}

\begin{enumerate}
	\item Робачевский А. Операционная система UNIX [Текст] -- 2010.
	\item Уорд Б. Внутреннее устройство Linux [Текст] -- 2016.
	\item Таненбаум Э. - Современные операционные системы [Текст] -- 2015.
	\item Керриск М. - Linux API. Исчерпывающее руководство [Текст] -- 2018.
	\item Работа с сигналами в Linux. [Электронный ресурс]:\\
		{\small\url{http://manpages.ubuntu.com/manpages/precise/ru/man7/signal.7.html}} 
\end{enumerate}

\end{document}
