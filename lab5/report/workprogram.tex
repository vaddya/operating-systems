\renewcommand{\labelenumii}{\theenumii}
\renewcommand{\theenumii}{\theenumi.\arabic{enumii}.}

Познакомиться с базовым набором IPC: сигналы, анонимные (неименованные) каналы, именованные каналы, очереди сообщений, семафоры, разделяемая память, сокеты.

\begin{enumerate}
	\item Сигналы:
		\begin{enumerate}
			\item Создать программу, позволяющую изменить диспозицию сигналов, а именно, установить:
				\begin{itemize}[itemsep=0.3em]
					\item обработчик пользовательских сигналов \code{SIGUSR1} и \code{SIGUSR2};
					\item реакцию по умолчанию на сигнал \code{SIGINT};
					\item игнорирование сигнала \code{SIGCHLD};
				\end{itemize}
			Повторить эксперимент для других сигналов, для процессов, порождаемых в разных в разных файлов, для потоков одного процесса, для потоков разных процессов.
			\item Создать программу, позволяющую продемонстрировать возможность отложенной обработки (временного блокирования) сигнала (например, \code{SIGINT}).
			
			Изменить обработчик сигнала так, чтобы из него производилась отправка другого сигнала.
			\item Проведите эксперимент, позволяющий определить возможность организации очереди для различных типов сигналов, обычных и реального времени, (более двух сигналов, для этого увеличьте «вложенность» вызовов обработчиков).
			
			Экспериментально подтвердите, что обработка равноприоритетных сигналов реального времени происходит в порядке FIFO.
			
			Опытным путем подтвердите наличие приоритетов сигналов реального времени.
		\end{enumerate}
	\item Каналы:
		\begin{enumerate}
			\item Организуем программу \code{pipe.c} так, чтобы процесс-родитель создавал неименованный канал, создавал потомка, закрывал канал на запись и записывал в произвольный текстовый файл считываемую из канала информацию. В функции процесса-потомка будет входить считывание данных из файла и запись их в канал.
			\item Создать клиент-серверное приложение, демонстрирующее дуплексную (двунаправленную) передачу информации двумя однонаправленными именованными каналами между клиентом и сервером.
		\end{enumerate}
	\item Очереди сообщений:
		\begin{enumerate}
			\item Создать клиент-серверное приложение, демонстрирующее передачу информации между процессами посредством очередей сообщений.
			\item К условиям предыдущей задачи добавляется наличие не единичного буфера, а буфера некоторого размера.
		\end{enumerate}
	\item Семафоры и разделяемая память:
		\begin{enumerate}
			\item Есть один процесс, выполняющий запись в разделяемую память и один процесс, выполняющий чтение из нее. Под чтением понимается извлечение данных из памяти. Программа должна обеспечить невозможность повторного чтения одних и тех же данных и невозможность перезаписи данных, т.е. новой записи, до тех пор, пока читатель не прочитает предыдущую.
			\item К условиям предыдущей задачи добавляется условие корректной работы нескольких читателей и нескольких писателей одновременно. Как и в предыдущем варианте под чтением понимается извлечение данных из памяти, т. е. одну порцию данных может прочитать только один читатель.
			\item К условиям предыдущей задачи добавляется наличие не единичного буфера, а буфера некоторого размера. Тип буфера (очередь, стек, кольцевой буфер) не имеет значения.
		\end{enumerate}
	\item Сокеты:
		\begin{enumerate}
			\item Сервер прослушивает заданный порт, при приходе нового соединения, создается новый поток для его обработки. Работа с клиентом организована как бесконечный цикл, в котором выполняется прием сообщения от клиента, вывод его на экран и пересылка обратно клиенту.
			\item Модифицировать, если необходимо, предложенное приложение и реализовать обмен сервера с множеством клиентов. Количество клиентов: 10, 100, 1000.
			\item Выполнить аналогичное взаимодействие на основе UDP.
		\end{enumerate}
\end{enumerate}
